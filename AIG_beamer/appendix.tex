\section{Appendix}

\begin{frame}{Gelfond-Lifschitz Reduct}
    \begin{block}{Definition of Reduct}
        For a logic program \(P\) and a set of literals \(M\), the \darkhighlight{reduct} \(P^M\) is defined as:
        \[
            P^M = \{head(r) \leftarrow pos(r) \mid r \in P, \mathhighlight{neg(r) \cap M = \emptyset}\}
        \]
        \begin{itemize}
            \item All rules where the negative body is not contradicted by \(M\).
            \item Removes default negation to simplify the program.
        \end{itemize}
    \end{block}
    \begin{block}{Answer Sets via Reduct}
        A set \(M\) is an \darkhighlight{answer set} of \(P\) if \(M\) is a minimal model of the reduct \(P^M\).
    \end{block}
\end{frame}

\begin{frame}{SI-Free Postulate Compliance}
    SI-free is not satisfied by any of the measures.
    \begin{exampleblock}{Example}
        Consider the logic program \(P = \{a \leftarrow \text{not } a, b., a \leftarrow \text{not }c., d \leftarrow \text{not }d., b., c., d.\}\), which has \(SI_{\min}(P) = \{\{a \leftarrow \text{not }a, b., b., c.\}\}\)
        and \(\alpha = d.\) and \(\beta = a \leftarrow \text{not }c.\) are in \(Free_{\text{SI}}(P)\).

        The measures evaluate to:
        \[
            \mathcal{I}_{\text{MSI}}(P) = 1,\quad \mathcal{I}_{\text{MSI}^\text{C}}(P) = \frac{1}{3},\quad \mathcal{I}_{\text{p}}(P) = 3.
        \]
        But if \(\alpha\) and \(\beta\) are removed, so \(SI_{\min}(P \backslash \{\alpha,\beta\}) = \{\{a \leftarrow \text{not }a, b., b.\}, \{d \leftarrow \text{not }d.\}\}\), they evaluate to:
        \[
            \mathcal{I}_{\text{MSI}}(P \backslash \{\alpha,\beta\}) = 2,\quad \mathcal{I}_{\text{MSI}^\text{C}}(P \backslash \{\alpha,\beta\}) = \frac{3}{2},\quad \mathcal{I}_{\text{p}}(P \backslash \{\alpha,\beta\}) = 3.
        \]
    \end{exampleblock}
\end{frame}
