\section{Introduction}

\begin{frame}{Handling Inconsistent Information}
    \begin{block}{Classical vs. Paraconsistent Logic}
        \begin{itemize}
            \item \darkhighlight{Classical logic} deems any inconsistency as catastrophic.
            \item \darkhighlight{Paraconsistent logic} allows reasoning with inconsistent knowledge bases.
        \end{itemize}
    \end{block}
    \begin{itemize}
        \item Inconsistency measures quantify the extent of inconsistencies in knowledge bases.
        \item Focus: Extending these measures to non-monotonic frameworks like ASP.
    \end{itemize}
\end{frame}

\begin{frame}{Inconsistency Measures}
    \begin{block}{Definition}
        An \darkhighlight{inconsistency measure} is a function $\mathcal{I}(\mathcal{K})$ that increases with the number of inconsistencies in a knowledge base $\mathcal{K}$. $\mathcal{I}(\mathcal{K}) = 0$ if $\mathcal{K}$ is consistent.
    \end{block}
    \begin{itemize}
        \item A popular approach is counting minimal inconsistent subsets: $\mathcal{I}_{\text{MI}}(\mathcal{K}) = \left| I_{\min}(\mathcal{K}) \right|$.
        \item Minimal inconsistent subsets represent atomic conflicts.
        \item [Example here]
    \end{itemize}
\end{frame}

\begin{frame}{Non-monotonic logics}
    \begin{block}{Definition}
        TODO
    \end{block}
\end{frame}

\section{Answer Set Programming}

\begin{frame}{Answer Set Programming (ASP)}
    \begin{block}{Logic Programming}
        \begin{itemize}
            \item ASP is a form of declarative programming for complex combinatorial problems.
            \item Logic programs contain disjunctions in the head and two kinds of negation:
                  \begin{itemize}
                      \item Strong negation ($\neg$)
                      \item Default negation (\texttt{not})
                  \end{itemize}
        \end{itemize}
    \end{block}
    TODO
\end{frame}

\begin{frame}{Answer Set Programming (ASP)}
    \begin{alertblock}{Rule Structure}
        A logic program $P$ consists of rules of the form:
        \[
            \darkmathhighlight{l_0 \lor \dots \lor l_k} \leftarrow \yellowmathhighlight{l_{k+1}, \dots, l_m}, \mathhighlight{\text{not } l_{m+1}, \dots, \text{not } l_n}
        \]
        Divided into:
        \begin{itemize}
            \item \darkhighlight{Head}: $head(r) = \{l_0, \dots, l_k\}$
            \item \yellowhighlight{Positive body}: $pos(r) = \{l_{k+1}, \dots, l_m\}$
            \item \highlight{Negative body}: $neg(r) = \{l_{m+1}, \dots, l_n\}$
        \end{itemize}
    \end{alertblock}
\end{frame}

\begin{frame}{Answer Set Programming (ASP)}
    \begin{block}{Reduct of a Logic Program}
        To extend answer sets to logic programs with negation, we define the \darkhighlight{reduct} of a program $P$ with respect to a set of literals $M$:
        \[
            P^M = \{head(r) \leftarrow pos(r) \mid r \in P, \mathhighlight{neg(r) \cap M = \emptyset}\}
        \]
    \end{block}
    \begin{block}{Answer Sets via Reduct}
        A set $M$ of literals is an \darkhighlight{answer set} of a logic program $P$ if:
        \begin{itemize}
            \item $M$ is an answer set (minimal model) of the \darkhighlight{reduct} $P^M$.
        \end{itemize}
    \end{block}
\end{frame}

\begin{frame}{Example}
    \begin{exampleblock}{Computing Answer Sets Using the Reduct}
        Consider the logic program:
        \[
            \begin{array}{lcl}
                a      & \leftarrow & \text{not } b \\
                b      & \leftarrow & \text{not } a \\
                c      & \leftarrow & a             \\
                \neg c & \leftarrow & b
            \end{array}
        \]
        Candidates are $M_1 = \{a, c\}$ and $M_2 = \{b, \neg c\}$\\
        Reducts:
        \[
            \begin{array}{lcl}
                P^{M_1} & = & \{a.\quad c \leftarrow a.\}      \\
                P^{M_2} & = & \{b.\quad \neg c \leftarrow b.\}
            \end{array}
        \]
        Both $M_1$ and $M_2$ are minimal models of their respective reducts; thus, both are \darkhighlight{answer sets} of $P$.
    \end{exampleblock}
\end{frame}

\section{Strong Inconsistency}

\begin{frame}{Strong Inconsistency in Non-Monotonic Logics}
    \begin{block}{Key Concept}
        Subsets of a knowledge base that remain insonsistent, even when additional information is added.
    \end{block}

    \begin{block}{Strongly $\mathcal{K}$-Inconsistent Subsets}
        $\mathcal{H} \subseteq \mathcal{K}$ is \darkhighlight{strongly $\mathcal{K}$-inconsistent} if all its supersets within $\mathcal{K}$ are inconsistent.

        Denoted as $SI(\mathcal{K})$.

        Minimal strongly $\mathcal{K}$-inconsistent subsets: $SI_{\min}(\mathcal{K})$.
    \end{block}

    \begin{exampleblock}{Resolving Inconsistency with Additional Information}
        $\mathcal{K} = \{b., \neg b \leftarrow \text{not } a.\}$

        Adding $a.$ to $\mathcal{K}$ makes it consistent.

        This shows that in non-monotonic logics, adding information can resolve conflicts.
    \end{exampleblock}
\end{frame}

\section{Measures Based on Minimal Inconsistent Subsets}

\begin{frame}{Inconsistency Measures for Non-Monotonic Logics}
    \begin{block}{Measure $\mathcal{I}_{\text{MSI}}$}
        \[
            \mathcal{I}_{\text{MSI}}(\mathcal{K}) = \left| SI_{\min}(\mathcal{K}) \right|
        \]
        Counts the number of minimal strongly inconsistent subsets.
    \end{block}
    \begin{block}{Measure $\mathcal{I}_{\text{MSI}}^\text{C}$}
        \[
            \mathcal{I}_{\text{MSI}}^\text{C}(\mathcal{K}) = \sum_{\mathcal{H} \in SI_{\min}(\mathcal{K})} \frac{1}{|\mathcal{H}|}
        \]
        Considers the size of each minimal inconsistent subset.
    \end{block}
\end{frame}

\begin{frame}{Inconsistency Measures for Non-Monotonic Logics}
    \begin{block}{Measure $\mathcal{I}_{\text{p}}$}
        \[
            \mathcal{I}_{\text{p}}(\mathcal{K}) = \left| \bigcup_{\mathcal{H} \in SI_{\min}(\mathcal{K})} \mathcal{H} \right|
        \]
        Counts the total number of problematic formulas.
    \end{block}
\end{frame}

\begin{frame}{Inconsistency Measures for Non-Monotonic Logics}
    \begin{exampleblock}{Calculating Inconsistency Measures}
        For $P = \{a \leftarrow \text{not } a, b.$, $a \leftarrow \text{not } c, \text{not } d.$, $b.$, $c.$, $d.\}$:

        \begin{itemize}
            \item $SI_{\min}(P) = \{\{a \leftarrow \text{not } a, b., b., c.\}, \{a \leftarrow \text{not } a, b., b., d.\}\}$
            \item $\mathcal{I}_{\text{MSI}}(P) = 2$
            \item $\mathcal{I}_{\text{MSI}}^\text{C}(P) = \frac{2}{3}$
            \item $\mathcal{I}_{\text{p}}(P) = 4$
        \end{itemize}
    \end{exampleblock}
\end{frame}

\section{Rationality Postulates}

\begin{frame}{Rationality Postulates}
    \begin{block}{Purpose}
        \begin{itemize}
            \item Ensure inconsistency measures behave reasonably.
            \item Provide guidelines for development and evaluation.
        \end{itemize}
    \end{block}
\end{frame}

\subsection{Basic Postulates}

\begin{frame}{Basic Postulates}
    \begin{block}{Consistency}
        For any knowledge base $\mathcal{K}$:
        \[
            \mathcal{I}(\mathcal{K}) = 0 \quad \text{iff} \quad \mathcal{K} \text{ is consistent}
        \]
    \end{block}
    \begin{block}{Strong Monotonicity}
        If $\mathcal{K}'$ \darkhighlight{preserves conflicts} of $\mathcal{K}$:
        \[
            \mathcal{I}(\mathcal{K}) \leq \mathcal{I}(\mathcal{K} \cup \mathcal{K}')
        \]
    \end{block}
\end{frame}

\begin{frame}{Independence Postulate}
    \begin{block}{Neutral Formulas}
        A formula $\alpha$ is \darkhighlight{neutral} if:
        \[
            \forall \mathcal{H} \subseteq \mathcal{K}: \mathcal{H} \in C(\mathcal{K}) \Leftrightarrow \mathcal{H} \cup \{\alpha\} \in C(\mathcal{K})
        \]
    \end{block}
    \begin{block}{Independence}
        If $\alpha$ is neutral:
        \[
            \mathcal{I}(\mathcal{K}) = \mathcal{I}(\mathcal{K} \backslash \{\alpha\})
        \]
    \end{block}
\end{frame}

\subsection{Extended Postulates}

\begin{frame}{Strong Equivalence}
    \begin{block}{Definition}
        Knowledge bases $\mathcal{K}$ and $\mathcal{K}'$ are \darkhighlight{strongly equivalent} ($\mathcal{K} \equiv_S \mathcal{K}'$) if:
        \[
            \text{ACC}(\mathcal{K} \cup \mathcal{G}) = \text{ACC}(\mathcal{K}' \cup \mathcal{G}) \quad \forall \mathcal{G}
        \]
    \end{block}
    \begin{block}{Postulate}
        If $\mathcal{K} \equiv_S \mathcal{K}'$:
        \[
            \mathcal{I}(\mathcal{K}) = \mathcal{I}(\mathcal{K}')
        \]
    \end{block}
\end{frame}

\begin{frame}{Separability}
    \begin{block}{Definition}
        If $SI_{\min}(\mathcal{K} \cup \mathcal{K}') = SI_{\min}(\mathcal{K}) \cup SI_{\min}(\mathcal{K}')$ and $SI_{\min}(\mathcal{K}) \cap SI_{\min}(\mathcal{K}') = \emptyset$:
        \[
            \mathcal{I}(\mathcal{K} \cup \mathcal{K}') = \mathcal{I}(\mathcal{K}) + \mathcal{I}(\mathcal{K}')
        \]
    \end{block}
\end{frame}

\begin{frame}{Compliance of Measures}
    \begin{block}{Proposition}
        Measures $\mathcal{I}_{\text{MSI}}$, $\mathcal{I}_{\text{MSI}}^\text{C}$, and $\mathcal{I}_{\text{p}}$ satisfy:

        \begin{itemize}
            \item Consistency
            \item Strong Monotonicity
            \item Independence
            \item FW-Strong Equivalence
            \item Strong Equivalence Replacement
            \item Strong Super-Additivity
        \end{itemize}

        Additionally, $\mathcal{I}_{\text{MSI}}$ and $\mathcal{I}_{\text{MSI}}^\text{C}$ satisfy Separability.
    \end{block}
\end{frame}

\section{Summary and Conclusion}

\begin{frame}{Summary}
    \begin{itemize}
        \item Addressed measuring inconsistencies in non-monotonic knowledge bases.
        \item Introduced \highlight{strong inconsistency} for non-monotonic logics.
        \item Applied strong inconsistency to extend inconsistency measures.
        \item Discussed rationality postulates for non-monotonic logics.
    \end{itemize}
\end{frame}

\begin{frame}{Conclusion}
    \begin{itemize}
        \item Provided an overview of inconsistencies in non-monotonic logics.
        \item Future work: Improve measures and compliance with rationality postulates.
        \item Applications: Repairing knowledge bases by adding formulas.
    \end{itemize}
\end{frame}

% Thank You slide
\begin{frame}
    \begin{center}
        \Huge{Thank You!}
    \end{center}
\end{frame}
