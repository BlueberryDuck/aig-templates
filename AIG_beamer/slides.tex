\section{Introduction}

\begin{frame}{Introduction to Inconsistent Information}
    \begin{itemize}
        \item \textbf{Knowledge Representation Challenge}
              \begin{itemize}
                  \item Handling inconsistent information is a fundamental challenge in knowledge representation and reasoning.
                  \item Inconsistencies can arise due to conflicting data, errors, or evolving knowledge.
              \end{itemize}
        \item \textbf{Classical Logic Limitations}
              \begin{itemize}
                  \item Classical logic deems any inconsistency as catastrophic, leading to \textbf{explosive} conclusions.
                  \item An inconsistent knowledge base is considered \textbf{useless} in classical semantics.
              \end{itemize}
        \item \textbf{Need for Alternative Approaches}
              \begin{itemize}
                  \item Not all inconsistencies have the same impact; some may be localized.
                  \item We require methods to handle and quantify inconsistencies without discarding the entire knowledge base.
              \end{itemize}
    \end{itemize}
\end{frame}

\begin{frame}{Classical vs. Paraconsistent Logic}
    \begin{block}{Classical Logic}
        \begin{itemize}
            \item \textbf{Principle of Explosion}: From a contradiction, anything can be inferred (\( \forall \phi, \psi: \{\phi, \neg \phi\} \vdash \psi \)).
            \item Treats any inconsistency as catastrophic.
            \item Inconsistent knowledge bases are deemed invalid.
        \end{itemize}
    \end{block}
    \begin{block}{Paraconsistent Logic}
        \begin{itemize}
            \item Allows reasoning in the presence of contradictions.
            \item Two inconsistent knowledge bases can lead to different conclusions.
            \item Inconsistencies are localized, preventing the principle of explosion.
        \end{itemize}
    \end{block}
\end{frame}

\begin{frame}{Inconsistency Measures}
    \begin{block}{Definition}
        An \darkhighlight{inconsistency measure} is a function \( \mathcal{I}(\mathcal{K}) \) that quantifies the extent and impact of inconsistencies within a knowledge base \( \mathcal{K} \).
    \end{block}
    \begin{itemize}
        \item \textbf{Purpose}:
              \begin{itemize}
                  \item Provides a quantitative metric rather than a binary consistent/inconsistent classification.
                  \item Helps in understanding the severity and localization of inconsistencies.
              \end{itemize}
        \item \textbf{Common Approach}:
              \begin{itemize}
                  \item Counting Minimal Inconsistent Subsets (MIS):
                        \[
                            \mathcal{I}_{\text{MI}}(\mathcal{K}) = \left| I_{\min}(\mathcal{K}) \right|
                        \]
                  \item \( I_{\min}(\mathcal{K}) \): Set of all minimal inconsistent subsets of \( \mathcal{K} \).
              \end{itemize}
        \item \textbf{Example}:
              \begin{itemize}
                  \item For \( \mathcal{K} = \{ a, \neg a, b \} \):
                  \item Minimal inconsistent subsets: \( I_{\min}(\mathcal{K}) = \{ \{ a, \neg a \} \} \)
                  \item Inconsistency measure: \( \mathcal{I}_{\text{MI}}(\mathcal{K}) = 1 \)
              \end{itemize}
    \end{itemize}
\end{frame}

\begin{frame}{Non-Monotonic Logics}
    \begin{block}{Definition}
        A \darkhighlight{non-monotonic logic} allows for the withdrawal of conclusions in the light of new, possibly conflicting, information.
    \end{block}
    \begin{itemize}
        \item \textbf{Contrast with Monotonic Logic}
              \begin{itemize}
                  \item In monotonic logic, adding new axioms cannot reduce the set of derivable conclusions.
                  \item Non-monotonic logic models the way human reasoning adapts to new information.
              \end{itemize}
        \item \textbf{Relevance to Inconsistency}
              \begin{itemize}
                  \item Non-monotonic logics can handle inconsistencies by revising conclusions.
                  \item They offer a framework for reasoning with incomplete or evolving knowledge.
              \end{itemize}
    \end{itemize}
\end{frame}

\begin{frame}{Example: The Penguin Paradox}
    \begin{exampleblock}{Example: The Penguin Paradox}
        \begin{itemize}
            \item \textbf{Initial Knowledge Base}:
                  \begin{itemize}
                      \item All birds can fly.
                      \item Penguins are birds.
                  \end{itemize}
            \item \textbf{Conclusion}:
                  \begin{itemize}
                      \item Therefore, penguins can fly.
                  \end{itemize}
            \item \textbf{Adding New Information}:
                  \begin{itemize}
                      \item Penguins cannot fly.
                  \end{itemize}
            \item \textbf{Non-Monotonic Adjustment}:
                  \begin{itemize}
                      \item Previous conclusion is retracted.
                      \item Knowledge base accommodates the exception.
                  \end{itemize}
        \end{itemize}
    \end{exampleblock}
\end{frame}

\section{Answer Set Programming}

\begin{frame}{Answer Set Programming (ASP)}
    \begin{alertblock}{Rule Structure}
        A logic program \(P\) consists of rules of the form:
        \[
            \darkmathhighlight{l_0 \lor \dots \lor l_k} \leftarrow \yellowmathhighlight{l_{k+1}, \dots, l_m}, \mathhighlight{\text{not }l_{m+1}, \dots, \text{not }l_n}
        \]
        Where:
        \begin{itemize}
            \item \darkhighlight{Head}: \(head(r) = \{l_0, \dots, l_k\}\)
            \item \yellowhighlight{Positive Body}: \(pos(r) = \{l_{k+1}, \dots, l_m\}\)
            \item \highlight{Negative Body}: \(neg(r) = \{l_{m+1}, \dots, l_n\}\)
        \end{itemize}
    \end{alertblock}
    \begin{itemize}
        \item Incorporates two kinds of negation:
              \begin{itemize}
                  \item \textbf{Strong Negation} ($\neg$): explicit denial of a fact.
                  \item \textbf{Default Negation} (\texttt{not}): assumption of falsehood in the absence of proof.
              \end{itemize}
    \end{itemize}
\end{frame}

\begin{frame}{Semantics of Logic Programs}
    \begin{block}{Models and Answer Sets}
        \begin{itemize}
            \item A \textbf{model} \(M\) of a logic program \(P\) satisfies all rules in \(P\).
            \item \(M\) is a model if for every rule \(r\):
                  \begin{itemize}
                      \item If \(pos(r) \subseteq M\) and \(neg(r) \cap M = \emptyset\), then \(head(r) \cap M \neq \emptyset\).
                  \end{itemize}
            \item An \textbf{answer set} is a minimal model of \(P\).
        \end{itemize}
    \end{block}
    \begin{block}{Handling Negation}
        \begin{itemize}
            \item The presence of default negation (\texttt{not}) requires special treatment.
            \item Use the \textbf{Gelfond-Lifschitz reduct} to eliminate default negation when computing answer sets.
        \end{itemize}
    \end{block}
\end{frame}

\begin{frame}{Example: The Penguin Paradox in ASP}
    \begin{exampleblock}{Example: The Penguin Paradox}
        \begin{itemize}
            \item \textbf{Facts}:
                  \[
                      \text{penguin(polly)}.
                  \]
            \item \textbf{Rules}:
                  \begin{align*}
                      \text{bird(X)}  & \leftarrow \text{penguin(X)}.                         \\
                      \text{flies(X)} & \leftarrow \text{bird(X)}, \ \text{not }\text{ab(X)}. \\
                      \text{ab(X)}    & \leftarrow \text{penguin(X)}.
                  \end{align*}
            \item \textbf{Interpretation}:
                  \begin{itemize}
                      \item Penguins are birds.
                      \item Birds typically fly unless they are abnormal (\texttt{ab(X)}).
                      \item Penguins are considered abnormal birds.
                  \end{itemize}
        \end{itemize}
    \end{exampleblock}
\end{frame}

\begin{frame}{Computing the Answer Set}
    \begin{block}{Candidate Answer Set \(M\)}
        \[
            M = \{\text{penguin(polly)},\ \text{bird(polly)},\ \text{ab(polly)}\}
        \]
    \end{block}
    \begin{block}{Constructing the Reduct \(P^M\)}
        \[
            P^M = \left\{
            \begin{array}{lcl}
                \text{penguin(polly)} &            &                       \\
                \text{bird(polly)}    & \leftarrow & \text{penguin(polly)} \\
                \text{ab(polly)}      & \leftarrow & \text{penguin(polly)}
            \end{array}
            \right\}
        \]
        \begin{itemize}
            \item The rule for \text{flies(polly)} is excluded since \(\text{not }\text{ab(polly)}\) is not satisfied.
        \end{itemize}
    \end{block}
\end{frame}

\section{Strong Inconsistency}

\begin{frame}{Strong Inconsistency in Non-Monotonic Logics}
    \begin{itemize}
        \item Adding information can resolve inconsistencies.
    \end{itemize}
    \begin{exampleblock}{Resolving Inconsistency}
        Knowledge Base \( \mathcal{K} = \{ b.,\ \neg b \leftarrow \text{not }a. \} \)
        \begin{itemize}
            \item \( \mathcal{K} \) is inconsistent since both \( b \) and \( \neg b \) can be derived.
            \item Adding \( a. \) resolves the inconsistency.
            \item Shows that some inconsistencies are not strong; they depend on missing information.
        \end{itemize}
    \end{exampleblock}
    \begin{itemize}
        \item Introduce \textbf{strong inconsistency} to better analyze inconsistencies.
    \end{itemize}
    \begin{block}{Definition}
        A subset \( \mathcal{H} \subseteq \mathcal{K} \) is \darkhighlight{strongly \( \mathcal{K} \)-inconsistent} if all its supersets within \( \mathcal{K} \) are inconsistent.
    \end{block}
\end{frame}

\section{Measures Based on Minimal Inconsistent Subsets}

\begin{frame}{Inconsistency Measures for Non-Monotonic Logics}
    \begin{itemize}
        \item Extending Classical Measures
              \begin{itemize}
                  \item Traditional measures count minimal inconsistent subsets (MIS).
                  \item Need to adapt these measures using strong inconsistency.
              \end{itemize}
        \item Counting Minimal Inconsistent Subsets (MIS):
              \[
                  \mathcal{I}_{\text{MI}}(\mathcal{K}) = \left| I_{\min}(\mathcal{K}) \right|
              \]
    \end{itemize}
    \begin{block}{Measure \( \mathcal{I}_{\text{MSI}} \)}
        \[
            \mathcal{I}_{\text{MSI}}(\mathcal{K}) = \left| SI_{\min}(\mathcal{K}) \right|
        \]
        Counts the number of minimal strongly inconsistent subsets.
    \end{block}
\end{frame}

\begin{frame}{Inconsistency Measures}
    \begin{block}{Measure \( \mathcal{I}_{\text{MSI}^\text{C}} \)}
        \[
            \mathcal{I}_{\text{MSI}^\text{C}}(\mathcal{K}) = \sum\limits_{\mathcal{H} \in SI_{\min}(\mathcal{K})} \frac{1}{|\mathcal{H}|}
        \]
        Considers the size of each minimal strongly inconsistent subset.
    \end{block}
    \begin{block}{Measure \( \mathcal{I}_{\text{p}} \)}
        \[
            \mathcal{I}_{\text{p}}(\mathcal{K}) = \left| \bigcup\limits_{\mathcal{H} \in SI_{\min}(\mathcal{K})} \mathcal{H} \right|
        \]
        Counts the total number of problematic formulas involved.
    \end{block}
\end{frame}

\begin{frame}{Example: Calculating Inconsistency Measures}
    \begin{exampleblock}{Example}
        \[
            P = \left\{ \begin{array}{l}
                a \leftarrow \text{not }a, b.              \\
                a \leftarrow \text{not }c,\ \text{not }d., \\
                b., c., d.
            \end{array} \right\}
        \]
        \textbf{Minimal Strongly Inconsistent Subsets}:
        \[
            SI_{\min}(P) = \left\{ \begin{array}{l}
                \{ a \leftarrow \text{not }a, b., b., c. \}, \\
                \{ a \leftarrow \text{not }a, b., b., d. \}
            \end{array} \right\}
        \]
        \textbf{Calculations}:
        \begin{itemize}
            \item \( \mathcal{I}_{\text{MSI}}(P) = 2 \)
            \item \( \mathcal{I}_{\text{MSI}^\text{C}}(P) = \frac{1}{3} + \frac{1}{3} = \frac{2}{3} \)
            \item \( \mathcal{I}_{\text{p}}(P) = 4 \)
        \end{itemize}
    \end{exampleblock}
\end{frame}

\section{Rationality Postulates}

\begin{frame}{Rationality Postulates}
    \begin{itemize}
        \item \textbf{Purpose}
              \begin{itemize}
                  \item Inconsistency measures quantify the \textbf{degree of inconsistency} in a knowledge base.
                  \item Not every function is suitable; we need criteria to ensure their \textbf{usefulness}.
                  \item These criteria are called \darkhighlight{rationality postulates}.
              \end{itemize}
        \item \textbf{Role in Inconsistency Measures}
              \begin{itemize}
                  \item Provide a foundation for \textbf{evaluating} and \textbf{comparing} inconsistency measures.
                  \item Ensure measures behave in a logically \textbf{consistent} and \textbf{expected} manner.
              \end{itemize}
    \end{itemize}
\end{frame}

\subsection{Basic Postulates}

\begin{frame}{Basic Rationality Postulates}
    % Selection based on Anthony Hunter and Sébastien Konieczny
    \begin{block}{Consistency Postulate}
        For any knowledge base \( \mathcal{K} \):
        \[
            \mathcal{I}(\mathcal{K}) = 0 \text{ if and only if } \mathcal{K} \text{ is consistent}
        \]
    \end{block}
\end{frame}

\begin{frame}{Issue with Monotonicity in Non-Monotonic Logics}
    \begin{alertblock}{Monotonicity}
        For knowledge bases \( \mathcal{K} \) and \( \mathcal{K}' \):
        \[
            \mathcal{I}(\mathcal{K}) \leq \mathcal{I}(\mathcal{K} \cup \mathcal{K}')
        \]
    \end{alertblock}
    \begin{itemize}
        \item \textbf{Intuition}:
              \begin{itemize}
                  \item Adding information should not \textbf{decrease} inconsistency.
                  \item More information potentially introduces more \textbf{conflicts}.
              \end{itemize}
    \end{itemize}
    \begin{itemize}
        \item \textbf{Non-Monotonic Behavior}
              \begin{itemize}
                  \item In non-monotonic logics, \textbf{adding information can resolve inconsistencies}.
                  \item Thus, \( \mathcal{I}(\mathcal{K} \cup \mathcal{K}') \) may be \textbf{less} than \( \mathcal{I}(\mathcal{K}) \).
              \end{itemize}
        \item \textbf{Implications}
              \begin{itemize}
                  \item Classical monotonicity postulate may not hold.
                  \item Need to \textbf{revise} the monotonicity postulate for non-monotonic settings.
              \end{itemize}
    \end{itemize}
\end{frame}

\begin{frame}{Revising Monotonicity: Strong Monotonicity}
    \begin{block}{Conflict Preservation}
        \textbf{Definition}: \( \mathcal{K}' \) \darkhighlight{preserves conflicts} of \( \mathcal{K} \) if:
        \[
            \forall \mathcal{H} \in SI(\mathcal{K}):\ \mathcal{H} \in SI(\mathcal{K} \cup \mathcal{K}')
        \]
    \end{block}
    \begin{block}{Strong Monotonicity Postulate}
        If \( \mathcal{K}' \) preserves conflicts of \( \mathcal{K} \):
        \[
            \mathcal{I}(\mathcal{K}) \leq \mathcal{I}(\mathcal{K} \cup \mathcal{K}')
        \]
    \end{block}
    \begin{itemize}
        \item \textbf{Intuition}:
              \begin{itemize}
                  \item If new information doesn't \textbf{resolve} existing conflicts, inconsistency should not decrease.
              \end{itemize}
    \end{itemize}
\end{frame}

\begin{frame}{Free Formulas}
    \begin{block}{Free Formulas}
        \textbf{Definition}: A formula \( \alpha \in \mathcal{K} \) is \darkhighlight{free} if:
        \[
            \alpha \notin \bigcup\limits_{\mathcal{H} \in SI_{\min}(\mathcal{K})} \mathcal{H}
        \]
    \end{block}
    \begin{block}{SI-Free Postulate}
        If \( \alpha \) is free in \( \mathcal{K} \):
        \[
            \mathcal{I}(\mathcal{K}) = \mathcal{I}(\mathcal{K} \backslash \{\alpha\})
        \]
    \end{block}
    \begin{alertblock}{Issue with SI-Free Postulate}
        \begin{itemize}
            \item In non-monotonic logics, removing a free formula can affect \( SI_{\min}(\mathcal{K}) \).
            \item May introduce new conflicts or alter existing ones.
        \end{itemize}
    \end{alertblock}
\end{frame}

\begin{frame}{Independence Postulate with Neutral Formulas}
    \begin{block}{Neutral Formulas}
        \textbf{Definition}: A formula \( \alpha \) is \darkhighlight{neutral} if:
        \[
            \forall \mathcal{H} \subseteq \mathcal{K}:\ \mathcal{H} \in C(\mathcal{K}) \Leftrightarrow \mathcal{H} \cup \{\alpha\} \in C(\mathcal{K})
        \]
    \end{block}
    \begin{block}{Independence Postulate}
        If \( \alpha \) is neutral in \( \mathcal{K} \):
        \[
            \mathcal{I}(\mathcal{K}) = \mathcal{I}(\mathcal{K} \backslash \{\alpha\})
        \]
    \end{block}
    \begin{itemize}
        \item \textbf{Benefit}:
              \begin{itemize}
                  \item Neutral formulas ensure that removing them doesn't affect inconsistency.
                  \item Addresses the issue with free formulas in non-monotonic settings.
              \end{itemize}
    \end{itemize}
\end{frame}

\subsection{Extended Postulates}

\begin{frame}{Strong Equivalence Postulates}
    \begin{block}{Strong Equivalence}
        Knowledge bases \( \mathcal{K} \) and \( \mathcal{K}' \) are \darkhighlight{strongly equivalent} if:
        \[
            \forall \mathcal{G}:\ \text{ACC}(\mathcal{K} \cup \mathcal{G}) = \text{ACC}(\mathcal{K}' \cup \mathcal{G})
        \]
    \end{block}
    \begin{block}{Strong Equivalence Postulate}
        If \( \mathcal{K} \equiv_S \mathcal{K}' \):
        \[
            \mathcal{I}(\mathcal{K}) = \mathcal{I}(\mathcal{K}')
        \]
    \end{block}
    \begin{itemize}
        \item \textbf{Intuition}:
              \begin{itemize}
                  %   \item ACC: The set of acceptable conclusions derived from the knowledge base (after possibly resolving inconsistencies).
                  \item Inconsistency measure should be \textbf{unaffected} under strong equivalence.
              \end{itemize}
    \end{itemize}
\end{frame}

\begin{frame}{FW-Strong Equivalence}
    \begin{block}{FW-Strong Equivalence}
        \(\mathcal{K}\) and \(\mathcal{K}'\) are \darkhighlight{formula-wise strongly equivalent} if:
        \[
            \exists\ \rho : \mathcal{K} \rightarrow \mathcal{K}'\ \text{bijective},\ \text{such that}\ \{\alpha\} \equiv_S \{\rho(\alpha)\}\ \forall \alpha \in \mathcal{K}
        \]
    \end{block}
    \begin{block}{FW-Strong Equivalence Postulate}
        If \(\mathcal{K} \equiv_{\alpha} \mathcal{K}'\):
        \[
            \mathcal{I}(\mathcal{K}) = \mathcal{I}(\mathcal{K}')
        \]
    \end{block}
\end{frame}

\begin{frame}{Strong Equivalence Replacement}
    \begin{block}{Strong Equivalence Replacement}
        If \(\{\alpha\} \equiv_S \{\alpha'\}\) and \(\alpha,\ \alpha' \notin \mathcal{K}\):
        \[
            \mathcal{I}(\mathcal{K} \cup \{\alpha\}) = \mathcal{I}(\mathcal{K} \cup \{\alpha'\})
        \]
    \end{block}
    \begin{itemize}
        \item Inconsistency measure should be insensitive to replacement by strongly equivalent formulas.
    \end{itemize}
\end{frame}

\begin{frame}{Separability and Strong Super-Additivity}
    \begin{block}{Separability Postulate}
        If:
        \begin{align*}
            SI_{\min}(\mathcal{K} \cup \mathcal{K}')            & = SI_{\min}(\mathcal{K}) \cup SI_{\min}(\mathcal{K}') \\
            SI_{\min}(\mathcal{K}) \cap SI_{\min}(\mathcal{K}') & = \emptyset
        \end{align*}
        Then:
        \[
            \mathcal{I}(\mathcal{K} \cup \mathcal{K}') = \mathcal{I}(\mathcal{K}) + \mathcal{I}(\mathcal{K}')
        \]
    \end{block}
    \begin{block}{Strong Super-Additivity Postulate}
        If \( \mathcal{K} \) and \( \mathcal{K}' \) preserve each other's conflicts and \( \mathcal{K} \cap \mathcal{K}' = \emptyset \):
        \[
            \mathcal{I}(\mathcal{K}) + \mathcal{I}(\mathcal{K}') \leq \mathcal{I}(\mathcal{K} \cup \mathcal{K}')
        \]
    \end{block}
\end{frame}

\begin{frame}{Summary of Rationality Postulates}
    \begin{itemize}
        \item \textbf{Basic Postulates}:
              \begin{itemize}
                  \item \textbf{Consistency}
                  \item \textbf{Strong Monotonicity}
                  \item \textbf{SI-Free and Independence}
              \end{itemize}
        \item \textbf{Extended Postulates}:
              \begin{itemize}
                  \item \textbf{Equivalence Postulates} (Strong Equivalence, FW-Strong Equivalence, and Strong Equivalence Replacement)
                  \item \textbf{Separability}
                  \item \textbf{Strong Super-Additivity}
              \end{itemize}
        \item \textbf{Purpose}:
              \begin{itemize}
                  \item Ensure inconsistency measures are \textbf{robust}, \textbf{meaningful}, and logically \textbf{sound}.
                  \item Provide guidelines for developing and evaluating inconsistency measures in \textbf{non-monotonic logics}.
              \end{itemize}
    \end{itemize}
\end{frame}

\section{Compliance of Measures}

\begin{frame}{Compliance of Measures}
    \centering
    \begin{tabular}{lccc}
        \toprule
                                       & \(\mathcal{I}_{\text{MSI}}\) & \(\mathcal{I}_{\text{MSI}^\text{C}}\) & \(\mathcal{I}_{\text{p}}\) \\
        \midrule
        Consistency                    & \ding{51}                    & \ding{51}                             & \ding{51}                  \\
        Strong monotonicity            & \ding{51}                    & \ding{51}                             & \ding{51}                  \\
        SI-Free                        & \ding{55}                    & \ding{55}                             & \ding{55}                  \\
        Independence                   & \ding{51}                    & \ding{51}                             & \ding{51}                  \\
        Strong Equivalence             & \ding{55}                    & \ding{55}                             & \ding{55}                  \\
        FW-Strong Equivalence          & \ding{51}                    & \ding{51}                             & \ding{51}                  \\
        Strong Equivalence Replacement & \ding{51}                    & \ding{51}                             & \ding{51}                  \\
        Separability                   & \ding{51}                    & \ding{51}                             & \ding{55}                  \\
        Strong Super-Additivity        & \ding{51}                    & \ding{51}                             & \ding{51}                  \\
        \bottomrule
    \end{tabular}
\end{frame}

\section{Summary and Conclusion}

\begin{frame}{Summary}
    \begin{itemize}
        \item \textbf{Handling Inconsistent Information}
              \begin{itemize}
                  \item Classical logic struggles with inconsistencies.
                  \item Inconsistency measures provide quantitative insights into conflicts.
              \end{itemize}
        \item \textbf{Answer Set Programming}
              \begin{itemize}
                  \item ASP as an example of non-monotonic logic.
                  \item Capable of representing and reasoning with inconsistent knowledge bases.
              \end{itemize}
        \item \textbf{Strong Inconsistency}
              \begin{itemize}
                  \item Defines conflicts that cannot be resolved by adding information.
                  \item Essential for measuring inconsistencies in non-monotonic logics.
              \end{itemize}
        \item \textbf{Inconsistency Measures}
              \begin{itemize}
                  \item Extended measures based on minimal inconsistent subsets using strong inconsistency.
                  \item Discussed measures: $\mathcal{I}_{\text{MSI}}$, $\mathcal{I}_{\text{MSI}^\text{C}}$, and $\mathcal{I}_{\text{p}}$.
              \end{itemize}
        \item \textbf{Rationality Postulates}
              \begin{itemize}
                  \item Reframed postulates to suit non-monotonic logics.
                  \item Evaluated measures' compliance with these postulates.
              \end{itemize}
    \end{itemize}
\end{frame}

\begin{frame}{Conclusion}
    \begin{itemize}
        \item \textbf{Limitations and Findings}
              \begin{itemize}
                  \item Measures generally align with rationality postulates.
                  \item Further refinement needed for SI-free and strong equivalence compliance.
              \end{itemize}
        \item \textbf{Future Directions}
              \begin{itemize}
                  \item Enhance inconsistency measures for better compliance.
                  \item Apply insights to practical applications like knowledge base repair.
                  \item Explore inconsistency measures in other non-monotonic frameworks.
              \end{itemize}
    \end{itemize}
\end{frame}
