In knowledge representation and reasoning, dealing with inconsistent information is a challenge because classical logic considers any inconsistency as rendering a knowledge base useless. However, paraconsistent logic allows different inconsistent knowledge bases to convey varying amounts of information, leading to the need for inconsistency measures, which are quantitative metrics that assess the extent and impact of inconsistencies within a knowledge base. This paper explores possible inconsistency measures within non-monotonic logics, focusing on logic programming with Answer Set Programming (ASP) as an example. By applying the concept of strong inconsistency, which identifies subsets of knowledge bases that remain inconsistent even when adding more information, to prior established measures based on minimal inconsistent subsets, the paper aims to make them applicable for non-monotonic logics. The paper further discusses revised rationality postulates, which are used to ensure theoretical soundness and practical applicability of such measures and assesses them against the previously established meausures.
