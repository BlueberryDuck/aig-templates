\section{Introduction}
text

\section{Measuring Strong Inconsistency}
\cite{ulbricht_measuring_2018}

"inconsistency measurements"
(Hunter and Konieczny 2004; Grant and Hunter 2006; Thimm 2016)
\cite{bertossi_approaches_2005} \cite{grant_measuring_2006}

"popular approach to measure inconsistency is to take the number of minimal inconsistent subsets"
(Hunter and Konieczny 2008)
\cite{hunter_measuring_2008}

"a wide variety of different inconsistency measures can be defined on top of that idea"
(Hunter and Konieczny 2008; Jabbour et al. 2016; Jabbour and Sais 2016)
\cite{hunter_measuring_2008} \cite{jabbour_mis_2016}

"Measuring inconsistency in nonmonotonic logics"
(Ulbricht, Thimm, and Brewka 2016; Brewka, Thimm, and Ulbricht 2017)
\cite{michael_measuring_2016} \cite{brewka_strong_2017}

"a refined notion of inconsistent subsets of a knowledge base K of a possibly nonmonotonic framework has been introduced, called strong K-inconsistency"
(Brewka, Thimm, and Ulbricht 2017)
\cite{brewka_strong_2017}

generalize existing inconsistency measures based on minimal inconsistent sets to arbitrary logics, which is the topic of the present paper
\cite{ulbricht_measuring_2018}

"rationality postulates, i.e., desirable properties that should hold for concrete approaches. There is a growing number of rationality postulates for inconsistency measurement but not every postulate is generally accepted"
(Besnard 2014)
\cite{ferme_revisiting_2014}

"This becomes apparent when considering the monotony postulate which is usually satisfied by classical inconsistency measures and demands \(I(K) \lesseqgtr I(K')\) whenever \(K \subseteq K'\) holds, i.e., the severity of inconsistency cannot be decreased by adding new information. However, in nonmonotonic frameworks, adding information may resolve conflicts. It is thus possible that \(K\) is inconsistent, while \(K'\) is not, so we would expect \(I(K') < I(K)\) for any reasonable measure I."
\cite{ulbricht_measuring_2018}

"we consider generalized versions of three measures based on minimal inconsistent sets" - Section 2
"we develop rationality postulates based on previous ones from the literature [...] most of them require refinements" - Section 3
\cite{ulbricht_measuring_2018}
