\section{Introduction}
In the context of knowledge representation and reasoning, there has always been a challenge of how to handle inconsistent information. Classical logic deems any inconsistency as catastrophic, rendering the affected knowledge base useless. Contrary to classical semantics, not all inconsistent knowledge bases contain the same amount of information, nor do they necessarily lead to the same conclusions. In paraconsistent logic, however, two inconsistent knowledge bases can entail different conclusions, indicating that they convey different information \cite{bertossi_approaches_2005}.

This shows the need for inconsistency measures, which means quantitative metrics that provide information about the extent and impact of conflicts within a knowledge base. A binary measure, commonly used for example in classical inference, only provides information on whether a knowledge base is consistent or inconsistent. Other logics that can work with inconsistent knowledge bases would profit from more fine-grained measures. One example of such logic is the quasi-classical logic \cite{grant_measuring_2006} which extends the first-order logic to enable the quantification of equality and more importantly inconsistency in a knowledge base by determining "degrees of inconsistency".

An inconsistency measure is a function \(\mathcal{I}\) applied to a knowledge base \(\mathcal{K}\) that increases the value as the number of inconsistencies in \(\mathcal{K}\) grows. A popular method for such a measure is to count the minimal inconsistent subsets, defined as \(I_{\min}(\mathcal{K})\), leading to the inconsistency measure \(\mathcal{I}_{\text{MI}}(\mathcal{K}) = \left| I_{\min}(\mathcal{K}) \right|\), which is zero if \(\mathcal{K}\) is consistent. This approach described by \cite{hunter_measuring_2008} uses the minimal number of formulas required to model the inconsistencies while also taking into account the proportion of the language that is touched by them. Many other inconsistency measures expand on that idea. For example, \cite{jabbour_mis_2016} uses minimal inconsistent subset partitions to be able to differentiate within the subsets how strong conflicts between them are. This means that not all subsets count equally towards the inconsistency score, which provides an even finer measure.

\section{Answer Set Programming}
In this paper, arbitrary and non-monotonic frameworks are to be discussed, with a focus on logic programming. Although the concept can be extended to cover a broader range of logics, this discussion will center on logic programming's foundational aspects. Answer set programming (ASP) \cite{gelfond_logic_2002}, a form of declarative programming initially intended for complex combinatorial search problems, exemplifies this focus.

Extended disjunctive databases, further only referred to as logic programs, contain a disjunction in the head of rules and use two kinds of negation. Strong negation (\(\neg\)) and default negation (not). A logic program \(P\) consists of a finite set of rules \(r\) which follow the form \(l_0 \lor \dots \lor l_k \leftarrow l_{k+1}, \dots, l_m, \text{not } l_{m+1}, \dots, \text{not }l_n\). This form can be split into three parts \(head(r) = \{l_0, \dots, l_k\}\), \(pos(r) = \{l_{k+1}, \dots, l_m\}\), and \(neg(r) = \{l_{m+1}, \dots, l_n\}\), resulting in an alternative notation \(head(r) \leftarrow pos(r), neg(r)\). If \(pos(r) \cup neg(r) = \emptyset\), then \(r\) is written "\(head(r)\)" instead of "\(head(r) \leftarrow\)".

Given this knowledge one can define an answer set as follows.

\begin{definition}[\cite{ulbricht_handling_2020}]
    Let \(P\) be a logic program over \(A\) such that \(neg(r) = \emptyset\) holds for each rule \(r \in P\). Then, a set \(M\) of literals is a model of \(P\) if for all \(r \in P\) the following is true: If \(pos(r) \subseteq M\), then \(head(r) \cap M \neq \emptyset\). If \(M\) is a model of \(P\) containing two complementary literals, then \(M\) is extended to \(M = Lit(A)\). A model \(M\) is minimal if for all proper subsets \(M'\) of \(M\), \(M'\) is not a model. A minimal model of \(P\) is called an answer set of \(P\).
\end{definition}

To extend the concept of answer sets to more complex logic programs, \cite{gelfond_classical_1991} defines the reduct of a program \(P\) with respect to a set \(M\), denoted as \(P^M\), which is essential for determining the answer sets of \(P\).

\begin{definition}
    Let \(P\) be a logic program and \(M\) be a set of literals. Let \[P^M = \{head(r) \leftarrow pos(r) | head(r) \leftarrow pos(r), neg(r) \in P, neg(r) \cap M = \emptyset\}\] be the reduct of \(P\) with regards to \(M\). \(M\) is an answer set of \(P\) if \(M\) is an answer set of \(P^M\).
\end{definition}

Logic programs can be modeled regarding a set \(A\) of propositional atoms as logic \(L_A^{ASP} = (\text{WF}_A^{ASP}, \text{BS}_A^{ASP}, \text{INC}_A^{ASP}, \text{ACC}_A^{ASP})\). \(\text{WF}_A^{ASP}\) is the set of all rules over \(A\), \(\text{BS}_A^{ASP}\) consists of the sets of literals over \(A\), \(\text{INC}_A^{ASP} = \{Lit(A)\}\), and \(\text{ACC}_A^{ASP}\) assigns the sets of all answers sets to its respective logic program.

\begin{example}
    Consider the following logic program \(P = \{a \lor b., a \leftarrow b., c \leftarrow \text{not }b., \neg c \leftarrow \text{not }b.\}\) This program has no answer set. Consider \{a\}, \{b\} and \{a,b\} as the candidates with
    \[\begin{array}{rrrrr}
            P^{\{a\}}:   & a \lor b. & a \leftarrow b. & c. & \neg c. \\
            P^{\{b\}}:   & a \lor b. & a \leftarrow b.                \\
            P^{\{a,b\}}: & a \lor b. & a \leftarrow b.
        \end{array}\]
    In this case \{a\} is not a model of \(P^{\{a\}}\), \{b\} is not a model of \(P^{\{b\}}\), and \{a,b\} is a model of \(P^{\{a,b\}}\) but is not minimal.
    \label{ex:program}
\end{example}

\section{Strong Inconsistency}
Consider the following notations with regards to a knowledge base \(\mathcal{K}\) in the context of propositional logic. \(I(\mathcal{K})\) denotes the set of all inconsistent subsets, \(I_{\min}(\mathcal{K})\) the set of all minimal inconsistent subsets, \(C(\mathcal{K})\) the set of all consistent, and \(C_{\max}(\mathcal{K})\) the set of all maximal \(\mathcal{K}\)-consistent subsets of \(\mathcal{K}\). In classical monotonic logics if a knowledge base \(\mathcal{K} \subseteq \mathcal{K}'\) is inconsistent then so is \(\mathcal{K}'\). Additionally, the hitting set duality establishes a relationship between minimal inconsistent sets and maximal consistent sets.

Let \(\mathcal{M}\) be a set of sets. \(\mathcal{H}\) is called a hitting set of \(\mathcal{M}\) if \(\mathcal{H}\) and all subsets \(M\) have a common intersection (\(\mathcal{H} \cap M \neq \emptyset | M \in \mathcal{M}\)). A minimal hitting set \(\mathcal{H}\) of \(\mathcal{M}\) further requires that no elements in \(\mathcal{H}\) can be dropped without losing its hitting set characteristic (\(\mathcal{H}' \subsetneq \mathcal{H}\) implies \(\mathcal{H}'\) is not a hitting set of \(\mathcal{M}\)). The following theorem, formulated within the context of propositional logic, is due to Reiter \cite{reiter_theory_1987}:

\begin{theorem}
    Let \(\mathcal{K}\) be a knowledge base in propositional logic, \(I_{\min}(\mathcal{K})\) be the set of minimal inconsistent subsets of \(\mathcal{K}\), and \(C_{\max}(\mathcal{K})\) be the set of maximal consistent subsets of \(\mathcal{K}\). \(\mathcal{H} \subseteq \mathcal{K}\) is a minimal hitting set of \(I_{\min}(\mathcal{K})\) if and only if \(\mathcal{K} \backslash \mathcal{H} \in C_{\max}(\mathcal{K})\).
\end{theorem}

In non-monotonic logics neither the statement that \(\mathcal{K}'\) must be inconsistent, nor the hitting set duality has to be true because additional information can resolve inconsistency in a knowledge base, which opens up the possibility of consistent knowledge bases to contain inconsistent subsets.

\begin{example}
    Consider the propositional signature \(A = \{a, b\}\) and the logic programs \(P = \{a., b., \neg b \leftarrow \text{not }a.\}\) and \(P' = \{b., \neg b \leftarrow \text{not }a.\}\). Then \(\text{ACC}_A^{ASP}(P) = \{\{a,b\}\}\) and \(\text{ACC}_A^{ASP}(P') = \{\}\). Observe that \(P\) is consistent while \(P'\) is inconsistent.
\end{example}

For this reason, a refined notion of inconsistent subsets of a knowledge base, specifically for non-monotonic logics, was defined by \cite{brewka_strong_2017}.
\begin{definition}
    A subset of a knowledge base \(\mathcal{K}\) is strongly \(\mathcal{K}\)-inconsistent if all its supersets within the knowledge base are inconsistent as well. The set of all strongly \(\mathcal{K}\)-inconsistent subsets of \(\mathcal{K}\) is denoted by \(SI(\mathcal{K})\). The set of all minimal strongly \(\mathcal{K}\)-inconsistent subsets of \(\mathcal{K}\) is denoted by \(SI_{\min}(\mathcal{K})\).
\end{definition}

Exploring the measurement of inconsistency in non-monotonic logics is now possible, a subject that has only been examined comparatively recently \cite{ulbricht_measuring_2018} \cite{brewka_strong_2019} \cite{ulbricht_handling_2020}.

\section{Measures based on Minimal Inconsistent Subsets}
The term strong inconsistency faithfully generalizes classical inconsistency to arbitrary logic, as shown by \cite{brewka_strong_2017}. The definition can still be applied to monotonic logics and further deems the existence of a strongly \(\mathcal{K}\)-inconsistent subset of \(\mathcal{K}\) as a necessary and sufficient condition for the inconsistency of \(\mathcal{K}\) itself.

\begin{theorem}
    Removing a hitting set of \(SI_{\min}(\mathcal{K})\) from \(\mathcal{K}\) results in a maximal consistent subset of \(\mathcal{K}\).
\end{theorem}

This section focuses on three measures of inconsistency which have been extended using strong inconsistency to apply to non-monotonic logics by \cite{ulbricht_handling_2020}.

A simple but popular approach to measuring inconsistency in classical logics is the use of minimal inconsistent subsets of a knowledge base \(\mathcal{K}\) because they represent "atomic conflicts" within it. This would be simply taking the value \(\left| I_{\min}(\mathcal{K}) \right|\). Applying strong inconsistency to this measure introduces the measure \(\mathcal{I}_{\text{MSI}}\) as follows.

\begin{definition}
    \(\mathcal{I}_{\text{MSI}}: 2^{\text{WF}} \rightarrow \mathbb{R}_{\geq 0}^{\infty} \text{ via } \mathcal{I}_{\text{MSI}}(\mathcal{K}) = \left| SI_{\min}(\mathcal{K}) \right|\)
\end{definition}

The restriction to limit the measure to non-negative numbers stems from its intended use. Intuitively an inconsistency measure \(\mathcal{I}(\mathcal{K}) = 0\) should mean the knowledge base measured is consistent, while \(\mathcal{I}(\mathcal{K}') < \mathcal{I}(\mathcal{K}'')\) should mean that \(\mathcal{K}'\) is "less" inconsistent than \(\mathcal{K}''\). The need to differentiate between consistent knowledge bases and thus using negative values is not given.

One extension of this approach is taking the size of a set \(\mathcal{H} \in SI_{\min}(\mathcal{K})\) into account. An inconsistency should be considered more severe, the smaller a minimal inconsistent subset is, which means that the contradiction can be represented with fewer formulas.

The lottery paradox \cite{kyburg_probability_1961} highlights this issue.
Assume a lottery with \(n\) tickets and statements \(t_1, \dots, t_n\) whether their respective ticket is the winner. As the lottery is fair and has exactly one winner (\(t_1 \lor \dots \lor t_n\)) and it is reasonable to assume a specific ticket \(t_i\) is likely to lose which can be applied to all tickets (\(\neg t_1, \dots, \neg t_n\)) it results in an inconsistent knowledge base \(\mathcal{K}_n = \{ t_1 \lor \dots \lor t_n, \neg t_1, \dots, \neg t_n \}\). Looking at two lotteries with \(n = 1\) and \(n = 1000000\) it doesn't seem as reasonable anymore to assume \(t_1\) loses in the first lottery compared to the latter. Although \(\mathcal{K}_1\) and \(\mathcal{K}_{1000000}\) are still both inconsistent \(\mathcal{K}_1\) seems a lot more unreasonable.

Coming back to the measure at hand, this intuition is not represented by \(\mathcal{I}_{\text{MSI}}\), as it provides \(\mathcal{I}_{\text{MSI}}(\mathcal{K}_n) = 1\) for all lotteries, no matter their size. To combat this \cite{ulbricht_handling_2020} proposes two further measures \(\mathcal{I}_{\text{MSI}^\text{C}}\) and \(\mathcal{I}_{\text{p}}\) by applying strong inconsistency to them.

The first measure \(\mathcal{I}_{\text{MSI}^\text{C}}\) is based on the measure \(\mathcal{I}_{\text{MI}^{\text{C}}}\) by \cite{hunter_measuring_2008} which considers the size of a minimal inconsistent set.

\begin{definition}
    \(\mathcal{I}_{\text{MSI}^\text{C}}: 2^{\text{WF}} \rightarrow \mathbb{R}_{\geq 0}^{\infty} \text{ via } \mathcal{I}_{\text{MSI}^\text{C}}(\mathcal{K}) = \sum\limits_{\mathcal{H} \in SI_{\min}(\mathcal{K})} \frac{1}{|\mathcal{H}|}\)
\end{definition}

The second measure \(\mathcal{I}_{\text{p}}\) is based on \cite{liu_measuring_2011} which considers the number of problematic formulas instead of the number of minimal inconsistent sets.

\begin{definition}
    \(\mathcal{I}_{\text{p}}: 2^{\text{WF}} \rightarrow \mathbb{R}_{\geq 0}^{\infty} \text{ via } \mathcal{I}_{\text{p}}(\mathcal{K}) = \left| \bigcup\limits_{\mathcal{H} \in SI_{\min}(\mathcal{K})} \mathcal{H} \right|\)
\end{definition}

\begin{example}
    Consider the logic program \(P = \{a \leftarrow \text{not }a, b., a \leftarrow \text{not }c, \text{not }d., b., c., d.\}\) which has \(SI_{\min}(P) = \{\{a \leftarrow \text{not }a, b., b., c.\}, \{a \leftarrow \text{not }a, b., b., d.\}\}\). The measures evaluate to \(\mathcal{I}_{\text{MSI}}(P) = 2\), \(\mathcal{I}_{\text{MSI}^\text{C}}(P) = \frac{2}{3}\), and \(\mathcal{I}_{\text{p}}(P) = 4\), because there are 2 minimal inconsistent subsets, the size of these subsets is 3 each, so \(\frac{1}{3}+\frac{1}{3} = \frac{2}{3}\), and there is a total of 4 unique problematic formulas in the union of these subsets.
\end{example}

\section{Rationality Postulates}
As described above, an inconsistency measure should indicate how much inconsistency a knowledge base carries. However, not just any function can be used for this purpose. A list of requirements is needed that an inconsistency measure must fulfill to be useful, these requirements are called rationality postulates.

There exist a number of these rationality postulates for inconsistency measures, which are revised and challenged in active discourse. For example, \cite{hunter_measure_2010} formulates a core set of rationality postulates, which are in turn revised by \cite{hameurlain_basic_2017} to ensure that inconsistency measures are robust to counterexamples, and alternative variants of this core set are proposed. This is done to refine rationality postulates and make them usable in different application contexts.

This is also the case in the context of inconsistency measures for non-monotonic logics, for which \cite{ulbricht_measuring_2018} and \cite{ulbricht_handling_2020} make several revisions to existing rationality postulates. The focal point here is on the monotony postulate which requires \(\mathcal{I}(\mathcal{K}) \leq \mathcal{I}(\mathcal{K}')\). In contrast to classical inconsistency measures, however, this postulate does not work for non-monotonic frameworks, as new information can resolve inconsistencies in a knowledge base, which means that \(\mathcal{I}(\mathcal{K}') < \mathcal{I}(\mathcal{K})\) is not only possible but also expected.

\subsection{Basic Postulates}
There are four basic postulates - Consistency, Monotony, Free Formula Independence, and Dominance - which a basic inconsistency measure should have, according to \cite{hunter_measure_2010}. Based on these, \cite{ulbricht_handling_2020} phrases the following basic postulates for non-monotonic logics.

\begin{description}
    \item[Consistency] For any knowledge base \(\mathcal{K} \subseteq \text{WF}\), \(\mathcal{I}(\mathcal{K}) = 0\) if and only if \(\mathcal{K}\) is consistent.
\end{description}

This requires an inconsistency measure to be able to distinguish between consistent and inconsistent knowledge bases and is true for non-monotonic logics without any modifications.

\begin{description}
    \item[\textit{Monotonicity}] If \(\mathcal{K}\) and \(\mathcal{K}'\) are knowledge bases, then \(\mathcal{I}(\mathcal{K}) \leq \mathcal{I}(\mathcal{K \cup \mathcal{K}'})\).
\end{description}

Monotonicity is a usually widely accepted postulate, which describes the intuition that a superset of a knowledge base cannot reduce inconsistency, only increase it, since more information is likely to be accompanied by more conflicts.

For non-monotonic logics, however, monotonicity does not make sense as new information might resolve conflicts in \(\mathcal{K}\). Monotonic behavior should only appear if \(\mathcal{K}'\) does not resolve any conflicts in \(\mathcal{K}\). If \(\mathcal{H} \subseteq \mathcal{K}\) is strongly inconsistent, so \(\mathcal{H} \in SI(\mathcal{K})\), then there should not exist a \(\mathcal{H}' \subseteq \mathcal{K}'\), such that \(\mathcal{H} \cup \mathcal{H}'\) is consistent. If there was, \(\mathcal{H}\) might not contribute to \(\mathcal{I}(\mathcal{K} \cup \mathcal{K}')\), which in turn questions the comparability of \(\mathcal{I}(\mathcal{K})\) and \(\mathcal{I}(\mathcal{K} \cup \mathcal{K}')\).

\begin{example}
    Recall the logic program \(P = \{a \lor b., a \leftarrow b., c \leftarrow \text{not }b., \neg c \leftarrow \text{not }b.\}\) from example \ref{ex:program}. The subprogram \(H \in SI_{\min}(P)\) consists of \(H = \{c \leftarrow \text{not }b., \neg c \leftarrow \text{not }b., a \leftarrow b.\}\). Now consider \(P' = \{b., d., \neg d.\}\) and \(H' = \{b.\} \subseteq P\) which resolves the conflict within P. \(H\) is not strongly (\(P \cup P'\))-inconsistent due to the consistent superprogram \(H \cup H'\). To be precise, \(P \cup \{b.\}\) is consistent but \(P'\) involves the conflict of \(d.\) and \(\neg d.\). In \(SI_{\min}(P \cup P') = \{\{d., \neg d.\}\}\) only the conflict within \(P'\) is represented which renders a comparison of \(P\) and \(P \cup P'\) unreasonable.
\end{example}

Acknowledging the issue with the classic monotonicity postulate results in the need for a rephrased monotonicity postulate that recognizes the preservation of conflicts as a condition for monotonicity in non-monotonic logics. Using the concept of strongly inconsistent subsets, preserving conflicts is defined as follows:

\begin{definition}\label{def:monotonic}
    Let \(\mathcal{K}\) and \(\mathcal{K}'\) be knowledge bases. \(\mathcal{K}'\) preserves conflicts of \(\mathcal{K}\) if \(\mathcal{H} \in SI(\mathcal{K} \cup \mathcal{K}')\) for any \(\mathcal{H} \in SI(\mathcal{K})\).
\end{definition}

Applying this concept of conflict preservation, \cite{ulbricht_handling_2020} provides the rephrased monotonicity postulate under the name strong monotonicity.

\begin{description}
    \item[Strong Monotonicity] If \(\mathcal{K}'\) preserves conflicts of \(\mathcal{K}\), then \(\mathcal{I}(\mathcal{K}) \leq \mathcal{I}(\mathcal{K \cup \mathcal{K}'})\).
\end{description}

Now coming to the free formula independence postulate which states that a free formula \(\alpha\) of a knowledge base \(\mathcal{K}\) should not increase the degree of inconsistency of \(\mathcal{K}\) regardless of its existence, so \(\mathcal{I}(\mathcal{K}) = \mathcal{I}(\mathcal{K} \backslash \{\alpha\})\) for \(\alpha \in Free(\mathcal{K})\). Free in this context then means that the formula \(\alpha\) is not part of a minimal inconsistent subset \(I_{\min}(\mathcal{K})\).

\begin{definition}
    Let \(\mathcal{K}\) be a monotonic knowledge base. A formula \(\alpha \in \mathcal{K}\) is called free if\\
    \(\alpha \in \mathcal{K} \backslash \bigcup\limits_{\mathcal{H} \in I_{\min}(\mathcal{K})} \mathcal{H}\).\\
    Let \(Free(\mathcal{K})\) be the set of all free formulas of \(\mathcal{K}\).
\end{definition}

To generalize the definition of free to non-monotonic logics as well, the concepts of hitting set duality and strong inconsistency can be applied. This results in the minimal inconsistent subset \(I_{\min}(\mathcal{K})\) to be replaced with \(SI_{\min}(\mathcal{K})\) and instead of \(\alpha\) having to appear in any minimal hitting set of \(SI_{\min}(\mathcal{K})\) one can assume \(\alpha\) to be present in all maximal consistent sets \(\mathcal{H} \in C_{\max}(\mathcal{K})\). This leads to the following definition:

\begin{definition}
    Let \(\mathcal{K}\) be a knowledge base. A formula \(\alpha \in \mathcal{K}\) is called free with respect to strong inconsistency (or SI-free or simply free if there is no risk of confusion) if\\
    \(\alpha \in \mathcal{K} \backslash \bigcup\limits_{\mathcal{H} \in SI_{\min}(\mathcal{K})} \mathcal{H} = \bigcap\limits_{\mathcal{H} \in C_{\max}(\mathcal{K})} \mathcal{H}\).\\
    Let \(Free_{\text{SI}}(\mathcal{K})\) be the set of all SI-free formulas of \(\mathcal{K}\).
\end{definition}

\begin{example}
    Recall the logic program \(P = \{a \lor b., a \leftarrow b., c \leftarrow \text{not }b., \neg c \leftarrow \text{not }b.\}\) from example \ref{ex:program}. As previously seen \(SI_{\min}(P) = \{c \leftarrow \text{not }b., \neg c \leftarrow \text{not }b., a \leftarrow b.\}\). To show that the formula \(\alpha = \{a \lor b.\}\) is free, consider
    \begin{align*}
        C_{\max}(P) = \{ & \{a \lor b., a \leftarrow b., c \leftarrow \text{not }b.\},                   \\
                         & \{a \lor b., a \leftarrow b., \neg c \leftarrow \text{not }b.\},              \\
                         & \{a \lor b., c \leftarrow \text{not }b., \neg c \leftarrow \text{not }b.\}\}.
    \end{align*}
    Since \(\alpha\) is the only formula occurring in all maximal consistent sets, it shows \(\bigcap_{H \in C_{\max}(P)} H = \{a \lor b\}\).
\end{example}

Given the similarities between \(Free(\mathcal{K})\) and \(Free_{\text{SI}}(\mathcal{K})\), a revised postulate for free formula independence is proposed by \cite{ulbricht_handling_2020}. This SI-Free postulate suggests that a free formula \(\alpha\) not only does not increase the degree of inconsistency but also does not introduce strongly inconsistent subsets.

\begin{description}
    \item[SI-Free] If \(\alpha \in Free_{\text{SI}}(\mathcal{K})\), then \(\mathcal{I}(\mathcal{K}) \leq \mathcal{I}(\mathcal{K} \backslash \{\alpha\})\).
\end{description}

One problem in using this postulate stems from the fact that ordinary inconsistency of a subset \(\mathcal{H} \subseteq \mathcal{K}\) only depends on \(\mathcal{H}\), whereas strong inconsistency relates to the whole knowledge base \(\mathcal{K}\). The removal of formulas in \(Free_{\text{SI}}(\mathcal{K})\) may affect the structure of \(SI_{\min}(\mathcal{K})\).

\begin{example}
    Consider the logic program \(P = \{a \leftarrow \text{not } a, b., a \leftarrow \text{not }c., d \leftarrow \text{not }d., b., c., d.\}\) which has \(SI_{\min}(P) = \{\{a \leftarrow \text{not }a,b., b., c.\}\}\). The formulas \(\alpha = \{d.\}\) and \(\beta = \{a \leftarrow \text{not }c.\}\) are in \(Free_{\text{SI}}(P)\). Although they are free, removing them affects \(SI_{\min}(P)\): Removing \(\alpha\) turns \(\{d \leftarrow \text{not }d.\}\) into a strongly inconsistent subset, while removing \(\beta\) makes \(\{c.\}\) irrelevant regarding the present conflict, so \(SI_{\min}(P \backslash \{\alpha, \beta\}) = \{\{a \leftarrow \text{not }a,b., b.\}, \{d \leftarrow \text{not }d.\}\}\).
\end{example}

The example shows that it is hard to predict what might happen when an SI-free formula is removed. Notably, \(SI_{\min}(\mathcal{K}) = SI_{\min}(\mathcal{K} \backslash \{\alpha\})\) is not true for all \(\alpha \in Free_{\text{SI}}(\mathcal{K})\), which shows that not even previously introduced \(\mathcal{I}_{\text{MSI}}\) satisfies this SI-free postulate.

In non-monotonic logics, an SI-free formula \(\alpha\) could resolve conflicts, which is why \cite{brewka_strong_2017} introduces the notion of neutral formulas to further strengthen the condition.

\begin{definition}\label{def:independence}
    Let \(\mathcal{K}\) be a knowledge base. A formula \(\alpha \in \mathcal{K}\) is called neutral if it satisfies\\
    \(\forall \mathcal{H} \subseteq \mathcal{K}: \mathcal{H} \in C(\mathcal{K}) \Leftrightarrow \mathcal{H} \cup \{\alpha\} \in C(\mathcal{K})\).\\
    Let \(Ntr(\mathcal{K})\) be the set of all neutral formulas of \(\mathcal{K}\).
\end{definition}

Neutral formulas do not make use of strong inconsistency, in contrast to SI-free formulas. Since neutral formulas do not depend as much on the structure of the knowledge base, they do not influence \(\mathcal{K}\) or \(SI_{\min}(\mathcal{K})\). Like SI-free formulas, a neutral formula \(\alpha \in \mathcal{K}\) does not increase the degree of inconsistency of \(\mathcal{K}\) but also does not resolve any conflicts either, which leads to the independence postulate.

\begin{description}
    \item[Independence] If \(\alpha \in Ntr(\mathcal{K})\), then \(\mathcal{I}(\mathcal{K}) = \mathcal{I}(\mathcal{K} \backslash \{\alpha\})\).
\end{description}

The only basic postulate from \cite{hunter_measure_2010} not mentioned yet is the dominance postulate. It tries to formulate the intuition that a formula carrying more information is also more likely to be involved in conflicts. It is highly contested, even for propositional settings, see \cite{ferme_revisiting_2014}, and \cite{ulbricht_measuring_2018} questions its use altogether for non-monotonic settings. Since for non-monotonic logics more information can even resolve conflicts in a knowledge base it is unfounded to consider such a formula to be more problematic.

This concludes the discussion on basic postulates.

\subsection{Extended Postulates}
For many approaches to inconsistency measurements, syntax is a decisive factor. One common example of this is the difference between the conjunction \(\{a \land b\}\) and two formulas \(\{a, b\}\). It is therefore desirable to be able to find equivalent formulas and knowledge bases for which a measure \(\mathcal{I}\) is robust with regards to the syntax of \(\mathcal{K}\). One of the postulates that enable this is the adjunction invariance by \cite{ferme_revisiting_2014} which formalizes this idea as \(\mathcal{I}(\mathcal{K} \cup \{a \land b\}) = \mathcal{I}(\mathcal{K} \cup \{a, b\})\).
To adjust this postulate for non-monotonic logics the term equivalence as in "two knowledge bases have the same model" is too weak. \cite{lifschitz_strongly_2001} developed the notion of strong equivalence to combat this problem, while \cite{brewka_strong_2019} added to this by generalizing it to an arbitrary logic.

\begin{definition}
    Let \(L = (\text{WF, BS, INC, ACC})\) be a logic. The knowledge bases \(\mathcal{K}\) and \(\mathcal{K}'\) are strongly equivalent, denoted by \(\mathcal{K} \equiv_S  \mathcal{K}'\), if \(\text{ACC}(\mathcal{K} \cup \mathcal{G}) = \text{ACC}(\mathcal{K}' \cup \mathcal{G})\) for each knowledge base \(\mathcal{G}\).
\end{definition}

Following this definition, a subset \(\mathcal{H}\) of a knowledge base \(\mathcal{K}\) can be swapped out with a strongly equivalent subset \(\mathcal{H}'\) without changing the inferences one can draw from \(\mathcal{K}\), which in turn means that it shouldn't affect the inconsistency they contribute to \(\mathcal{K}\). Applied to the whole knowledge base, this means that two strongly equivalent knowledge bases should have the same inconsistency measure.

\begin{description}
    \item[Strong Equivalence] If \(\mathcal{K} \equiv_S \mathcal{K}'\), then \(\mathcal{I}(\mathcal{K}) = \mathcal{I}(\mathcal{K}')\).
\end{description}

In some cases, the above postulate is too strong resulting in \(\mathcal{K} \equiv_S \mathcal{K}'\) for any two inconsistent knowledge bases, which is why a more fine-grained notion of equivalence is needed. To achieve this a similar approach as in \cite{thimm_inconsistency_2013} can be used to define:

\begin{definition}\label{def:fwequivalent}
    Let \(\mathcal{K}\) and \(\mathcal{K}'\) be two knowledge bases. \(\mathcal{K}\) and \(\mathcal{K}'\) are called formula-wise strongly equivalent, denoted by \(\mathcal{K} \equiv_{\alpha} \mathcal{K}'\), if there is a bijection \(\rho : \mathcal{K} \rightarrow \mathcal{K}'\) such that \(\{\alpha\} \equiv_S \{\rho(\alpha)\}\) holds for all \(\alpha \in \mathcal{K}\).
\end{definition}

Applying this notion to the previously defined strong equivalence, a new rationality postulate can be formed, which considers a formula-wise equivalence (FW=formula-wise):

\begin{description}
    \item[FW-Strong Equivalence] If \(\mathcal{K} \equiv_{\alpha} \mathcal{K}'\), then \(\mathcal{I}(\mathcal{K}) = \mathcal{I}(\mathcal{K}')\).
\end{description}

\cite{ulbricht_handling_2020} further refines this notion by considering the replacement of a formula \(\alpha\) with a strongly equivalent formula \(\alpha'\), resulting in:

\begin{description}
    \item[Strong Equivalence Replacement] If \(\{\alpha\} \equiv_S \{\alpha'\}\) and \(\alpha \notin \mathcal{K}\) as well as \(\alpha' \notin \mathcal{K}\), then \(\mathcal{I}(\mathcal{K} \cup \{\alpha\}) = \mathcal{I}(\mathcal{K} \cup \{\alpha'\})\).
\end{description}

Addressing the modularization of a knowledge base \(\mathcal{K}\) \cite{ulbricht_measuring_2018} considers two more postulates. The separability postulate by \cite{hunter_measure_2010} can be easily applied to strongly inconsistent subsets.

\begin{description}
    \item[Separability] If \(SI_{\min}(\mathcal{K} \cup \mathcal{K}') = SI_{\min}(\mathcal{K}) \cup SI_{\min}(\mathcal{K}')\) and \(SI_{\min}(\mathcal{K}) \cap SI_{\min}(\mathcal{K}') = \emptyset\), then \(\mathcal{I}(\mathcal{K} \cup \mathcal{K}') = \mathcal{I}(\mathcal{K}) + \mathcal{I}(\mathcal{K}')\).
\end{description}

This postulate ensures that the inconsistency values of \(\mathcal{K}\) and \(\mathcal{K}'\) should sum up to the value of their union because their respective conflicts should be independent of one another.

Lastly, the super-additivity postulate by \cite{thimm_measuring_2009} can be adjusted to the notion of strong monotonicity, by accounting for the fact that adding information might resolve conflicts for non-monotonic logics. This can be achieved by adding conflict preservation as an additional condition.

\begin{description}
    \item[Strong Super-Additivity] If \(\mathcal{K}\) and \(\mathcal{K}'\) preserve each others's conflicts and \(\mathcal{K} \cap \mathcal{K}' = \emptyset\), then \(\mathcal{I}(\mathcal{K}) + \mathcal{I}(\mathcal{K}') \leq \mathcal{I}(\mathcal{K} \cup \mathcal{K}')\).
\end{description}

This concludes the discussion of rationality postulates with a generalization of monotonicity for non-monotonic logics. To evaluate the effectiveness of these postulates, the compliance regarding the proposed measures is considered:

\begin{proposition}[\cite{ulbricht_measuring_2018}]
    The measures \(\mathcal{I}_{\text{MSI}}\), \(\mathcal{I}_{\text{MSI}^\text{C}}\) and \(\mathcal{I}_{\text{p}}\) satisfy consistency, strong monotonicity, independence, FW-strong equivalence, strong equivalent replacement, and strong super-additivity. The measures \(\mathcal{I}_{\text{MSI}}\) and \(\mathcal{I}_{\text{MSI}^\text{C}}\) also satisfy seperability.
\end{proposition}

These results highlight the strengths and limitations of the proposed measures in measuring inconsistencies in non-monotonic logics, specifically noting the lack of SI-free compliance.

\section{Summary and Conclusion}
tbd