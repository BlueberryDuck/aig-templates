\section{Non-Monotonic Logics}
The term "non-monotonic logic" may at first evoke discomfort for many, as it differentiates from classical logic and appears complex. But reality often presents scenarios that cannot be fully described by classical logic, necessitating the exploration of non-monotonic logic.\\
In classical logic systems, reasoning processes require disciplined and formally deduced thoughts. Everything is defined and relations can be formalized. Based on these assumptions, irrefutable conclusions can be drawn from a given knowledge base. This deductive process underpins the monotonicity of classical logic.\\
Real scenarios in contrast are often characterized by unexpected events and complex relations, which elude formalization. Descriptions of predicates like "small" require context; an elephant may be small compared to a mountain but not to mice. Additionally, are objects and systems subject to constant temporal changes, highlighting the limitations of classical logic in representing dynamic real-world knowledge.\\

\subsection{Introduction to Non-Monotonic Logics}
text\\
"The purpose of non-monotonic inference rules is not to add certain knowledge where there is none, but rather to guide the selection of tentatively held beliefs in the hope that fruitful investigations and good guesses will result. This means that one should not \textit{a priori} expect non-monotonic rules to derive valid conclusions independent of the monotonic rules. Rather one should expect to be led to a set of beliefs which while perhaps eventually shown incorrect will meanwhile coherently guide investigations."\\
\cite[p. 46]{mcdermott_non-monotonic_1980}

\subsection{Understanding Inconsistencies in Logic Systems}
text

\subsection{Challenges Specific to Non-Monotonic Logics}
text

\section{Measuring Inconsistency}
Analyze the distribution of inconsistency within a knowledge base using the Shapely measure \cite{hunter_measure_2010} in the classical case. Measures based on minimal inconsistent sets of a knowledge base \(\mathcal{K}\) \cite{jabbour_mis_2016} were proposed by \cite{ulbricht_measuring_2018}.

\subsection{Measures of Inconsistency in Non-Monotonic logics}
text

\subsection{Case Studies and Applications}
text

\subsection{Future Directions}
text
