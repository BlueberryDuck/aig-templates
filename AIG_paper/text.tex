\section{Non-Monotonic Logics}
The term "non-monotonic logic" may initially evoke discomfort due to the complexities and deviations from classical logic.
However, the real world often presents scenarios that classical logic cannot adequately address, necessitating the exploration of non-monotonic logic.
In classical logic, the reasoning processes require thoughts to be disciplined and conform to formal rules.
Everything has a distinct meaning, and relationships can be formalized. Based on these assumptions, one can draw irrefutable conclusions from given knowledge using deduction.
This deductive process underpins the monotonicity of classical logic.

In contrast, real-world scenarios are characterized by unexpected events and complex relationships that often resist formalization.
Descriptions of predicates such as "small" are context-dependent; for instance, an elephant is small compared to a mountain but not to a mouse.
Additionally, temporal changes affect the state of objects and systems continuously, highlighting the limitations of classical logic in representing dynamic real-world knowledge.

Human knowledge is inherently incomplete, and everyday life involves drawing conclusions based on incomplete information.
Errors in assumptions and conclusions are common, yet do not collapse the entire worldview.

Several factors are listed by \cite{beierle_methoden_2019} that contribute to this knowledge imperfection:
\begin{itemize}
    \item Some information is unknown due to lack of learning, forgetfulness, general unavailability, or inaccessibility.
          Time constraints also necessitate decisions based on limited information.
    \item Describing real situations completely in a logical sense is impossible.
          Listing all conceivable exceptions to a rule is impractical, and checking all details is time-consuming.
    \item Abstraction from seemingly unimportant aspects to focus on essentials often leads to incomplete information, which can be both sensible and intentional.
    \item Situational characteristics can be overlooked or misinterpreted.
    \item Future events can only be speculated upon.
    \item Natural language is context-dependent and rarely entirely clear, requiring interpretation and posing a risk of misunderstandings.
\end{itemize}

Despite these challenges, humans handle incomplete information successfully for decision-making and navigation.
This capability is considered a significant intelligence achievement, demonstrating the effectiveness of non-monotonic reasoning.
Humans excel in non-monotonic reasoning, which should be considered a standard form of logic rather than a peripheral one.
However, non-monotonic logic often appears difficult and esoteric due to its complex nature.

\subsection{Introduction to Non-Monotonic Logics}
"The purpose of non-monotonic inference rules is not to add certain knowledge where there is none, but rather to guide the selection of tentatively held beliefs in the hope that fruitful investigations and good guesses will result.
This means that one should not \textit{a priori} expect non-monotonic rules to derive valid conclusions independent of the monotonic rules.
Rather one should expect to be led to a set of beliefs which while perhaps eventually shown incorrect will meanwhile coherently guide investigations."\\
\cite[p. 46]{mcdermott_non-monotonic_1980}

\subsection{Understanding Inconsistencies in Logic Systems}
text

\subsection{Challenges Specific to Non-Monotonic Logics}
text

\section{Measuring Inconsistency}
Analyze the distribution of inconsistency within a knowledge base using the Shapely measure \cite{hunter_measure_2010} in the classical case.
Measures based on minimal inconsistent sets of a knowledge base \(\mathcal{K}\) \cite{jabbour_mis_2016} were proposed by \cite{ulbricht_measuring_2018}.

\subsection{Measures of Inconsistency in Non-Monotonic logics}
text

\subsection{Case Studies and Applications}
text

\subsection{Future Directions}
text
