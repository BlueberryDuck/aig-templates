\section{Introduction}
text

\section{Inconsistency Measures}
Not all inconsistent knowledge bases contain no information at all or the same amount of, just because they are inconsistent as a whole. This can be deduced as two inconsistent knowledge bases can provide different conclusions, implying they contain different information \cite{bertossi_approaches_2005}.

A binary measure, commonly used for example in classical inference, only provides information on whether a knowledge base is consistent or inconsistent. Other logics that can work with inconsistent knowledge bases would profit from more fine-grained measures.
One example of such logic is the quasi-classical logic \cite{grant_measuring_2006} which extends the first-order logic to enable the quantification of equality and more importantly inconsistency in a knowledge base by determining "degrees of inconsistency".

The fundamental idea of an inconsistency measure is to provide a function \(\mathcal{I}\) that when applied to a knowledge base \(\mathcal{K}\) is growing in number the more inconsistencies there are in \(\mathcal{K}\). One of the most popular approaches for such a measure is using the number of minimal inconsistent subsets \cite{hunter_measuring_2008}. This approach "counts" the minimal number of formulas needed to model the inconsistency of the entire set, while also taking into account the proportion of the language that is touched by the inconsistency.

So when using a set of all minimal inconsistent subsets of a knowledge base \(I_{\min}(\mathcal{K})\) one can define an inconsistency measure as \(\mathcal{I}_{\text{MI}}(\mathcal{K}) = \left| I_{\min}(\mathcal{K}) \right|\). This simple approach comes with the advantage of already meeting a number of the basics that an inconsistency measure needs to fulfill, most notably \(\mathcal{I}_{\text{MI}}(\mathcal{K}) = 0\) if \(\mathcal{K}\) is consistent.

Many other inconsistency measures expand on that idea. For example, \cite{jabbour_mis_2016} uses minimal inconsistent subset partitions to be able to differentiate within the subsets how strong conflicts between them are. This means that not all subsets count equally towards the inconsistency score, which provides an even finer measure.

\section{Definition Weakly Monotonic and Non-Monotonic Logics}
 [Both terms are used without definition in a later chapter]

\section{Definition Answer Set Programming}
 [Possibly defined here for continuous use as an example throughout the rest of the paper]
\\
"Some examples of such formalisms are, e.g. [...] answer set programming \cite{gelfond_logic_2002} [...]"
\\
see Example 2.2 in \cite{ulbricht_measuring_2018}

\section{Strong Inconsistency}
Let \(\mathcal{K}\) be the knowledge base of an explicit logic.

    [Example here]

\(I(\mathcal{K})\) denotes the set of all inconsistent subsets, \(I_{\min}(\mathcal{K})\) the set of all minimal inconsistent subsets, \(C(\mathcal{K})\) the set of all consistent, and \(C_{\max}(\mathcal{K})\) the set of all maximal \(\mathcal{K}\)-consistent subsets of \(\mathcal{K}\). In weakly monotonic logics if a knowledge base \(\mathcal{K} \subseteq \mathcal{K}'\) is inconsistent then so is \(\mathcal{K}'\). Additionally, a specific duality between minimal inconsistent and maximal consistent sets holds true.

Let \(\mathcal{M}\) be a set of sets. We call \(\mathcal{H}\) a hitting set of \(\mathcal{M}\) if \(\mathcal{H}\) and all subsets \(M\) have a common intersection (\(\mathcal{H} \cap M \neq \emptyset | M \in \mathcal{M}\)). A minimal hitting set \(\mathcal{H}\) of \(\mathcal{M}\) further requires that no elements in \(\mathcal{H}\) can be dropped without losing its hitting set characteristic (\(\mathcal{H}' \subsetneq \mathcal{H}\) implies \(\mathcal{H}'\) is not a hitting set of \(\mathcal{M}\)).

\begin{theorem}
    TODO:
    The MinHS duality \cite{reiter_theory_1987} implies \(\mathcal{H}\) is a minimal hitting set of \(I_{\min}(\mathcal{K})\) only if the knowledge base without the elements of \(\mathcal{H}\) forms the maximal consistent subset (\(\mathcal{K} \backslash \mathcal{H} \in C_{\max}(\mathcal{K})\)).
\end{theorem}

In non-monotonic logics neither the statement that \(\mathcal{K}'\) must be inconsistent, nor the hitting set duality has to be true because additional information can resolve inconsistency in a knowledge base, which opens up the possibility of consistent knowledge bases to contain inconsistent subsets.

    [Example here]

For this reason, a refined notion of inconsistent subsets of a knowledge base was defined by \cite{brewka_strong_2017}.
\begin{definition}
    TODO:
    Following this definition, a subset of a knowledge base \(\mathcal{K}\) is strongly \(\mathcal{K}\)-inconsistent if all its supersets within the knowledge base are inconsistent as well. This introduces the denotations \(SI(K)\) as the set of all strongly \(\mathcal{K}\)-inconsistent subsets of \(\mathcal{K}\) and \(SI_{\min}(\mathcal{K})\) as the set of all minimal strongly \(\mathcal{K}\)-inconsistent subsets of \(\mathcal{K}\).
\end{definition}

This opens up the possibility of exploring the measurement of inconsistency in non-monotonic logics, a subject that has only been examined comparatively recently \cite{ulbricht_measuring_2018} \cite{brewka_strong_2019} \cite{ulbricht_handling_2020}.

\section{Three Measures based on Minimal Inconsistent Sets}
The term strong inconsistency faithfully generalizes classical inconsistency to arbitrary logic, as shown by \cite{brewka_strong_2017}. The definition can still be applied to monotonic logics and further deems the existence of a strongly \(\mathcal{K}\)-inconsistent subset of \(\mathcal{K}\) as a necessary and sufficient condition for the inconsistency of \(\mathcal{K}\) itself.

\begin{theorem}
    TODO:
    Additionally, the hitting set duality now also applies to any logic as well, meaning removing a hitting set of \(SI_{\min}(\mathcal{K})\) from \(\mathcal{K}\) results in a maximal consistent subset of \(\mathcal{K}\).
\end{theorem}

This section focuses on three measures of inconsistency which have been extended using strong inconsistency to apply to non-monotonic logics by \cite{ulbricht_handling_2020}.

Let \(L\) be a [example here] logic. A simple but popular approach to measuring inconsistency in classical logics is the use of minimal inconsistent subsets of a knowledge base \(\mathcal{K}\) because they represent "atomic conflicts" within it. This would be simply taking the value \(\left| I_{\min}(\mathcal{K}) \right|\). Applying strong inconsistency to this measure introduces the measure \(\mathcal{I}_{\text{MSI}}\) as follows.

\begin{definition}
    \(\mathcal{I}_{\text{MSI}}: 2^{\text{WF}} \rightarrow \mathbb{R}_{\geq 0}^{\infty} \text{ via } \mathcal{I}_{\text{MSI}}(\mathcal{K}) = \left| SI_{\min}(\mathcal{K}) \right|\)
\end{definition}

The restriction to limit the measure to non-negative numbers stems from its intended use. Intuitively an inconsistency measure \(\mathcal{I}(\mathcal{K}) = 0\) should mean the knowledge base measured is consistent, while \(\mathcal{I}(\mathcal{K}') < \mathcal{I}(\mathcal{K}'')\) should mean that \(\mathcal{K}'\) is "less" inconsistent than \(\mathcal{K}''\). The need to differentiate between consistent knowledge bases and thus using negative values is not given.

One extension of this approach is taking the size of a set \(\mathcal{H} \in SI_{\min}(\mathcal{K})\) into account. An inconsistency should be considered more severe, the smaller a minimal inconsistent subset is, which means that the contradiction can be represented with fewer formulas.

The lottery paradox \cite{kyburg_probability_1961} highlights this issue.
Assume a lottery with \(n\) tickets and statements \(t_1, ..., t_n\) whether their respective ticket is the winner. As the lottery is fair and has exactly one winner (\(t_1 \lor ... \lor t_n\)) and it is reasonable to assume a specific ticket \(t_i\) is likely to lose which can be applied to all tickets (\(\neg t_1, ..., \neg t_n\)) it results in an inconsistent knowledgebase \(\mathcal{K}_n = \{ t_1 \lor ... \lor t_n, \neg t_1, ..., \neg t_n \}\). Looking at two lotteries with \(n = 1\) and \(n = 1000000\) it doesn't seem as reasonable anymore to assume \(t_1\) loses in the first lottery compared to the latter. Although \(\mathcal{K}_1\) and \(\mathcal{K}_{1000000}\) are still both inconsistent \(\mathcal{K}_1\) seems a lot more unreasonable.

Coming back to the measure at hand, this intuition is not represented by \(\mathcal{I}_{\text{MSI}}\), as it provides \(\mathcal{I}_{\text{MSI}}(\mathcal{K}_n) = 1\) for all lotteries, no matter their size. To combat this \cite{ulbricht_handling_2020} proposes two further measures \(\mathcal{I}_{\text{MSI}^\text{C}}\) and \(\mathcal{I}_{\text{p}}\) by applying strong inconsistency to them.

The first measure \(\mathcal{I}_{\text{MSI}^\text{C}}\) is based on the measure \(\mathcal{I}_{\text{MI}^{\text{C}}}\) by \cite{hunter_measuring_2008} which considers the size of a minimal inconsistent set.

\begin{definition}
    \(\mathcal{I}_{\text{MSI}^\text{C}}: 2^{\text{WF}} \rightarrow \mathbb{R}_{\geq 0}^{\infty} \text{ via } \mathcal{I}_{\text{MSI}^\text{C}}(\mathcal{K}) = \sum\limits_{\mathcal{H} \in SI_{\min}(\mathcal{K})} \frac{1}{|\mathcal{H}|}\)
\end{definition}

The second measure \(\mathcal{I}_{\text{p}}\) is based on \cite{liu_measuring_2011} which considers the number of problematic formulas instead of the number of minimal inconsistent sets.

\begin{definition}
    \(\mathcal{I}_{\text{p}}: 2^{\text{WF}} \rightarrow \mathbb{R}_{\geq 0}^{\infty} \text{ via } \mathcal{I}_{\text{p}}(\mathcal{K}) = \left| \bigcup\limits_{\mathcal{H} \in SI_{\min}(\mathcal{K})} \mathcal{H} \right|\)
\end{definition}

\section{Rationality Postulates}
As described above, an inconsistency measure should indicate how much inconsistency a knowledge base carries. However, not just any function can be used for this purpose. A list of requirements is needed that an inconsistency measure must fulfill to be useful, these requirements are called rationality postulates.

There exist a number of these rationality postulates for inconsistency measures, which are revised and challenged in active discourse. For example, \cite{hunter_measure_2010} formulates a core set of rationality postulates, which are in turn revised by \cite{hameurlain_basic_2017} to ensure that inconsistency measures are robust to counterexamples, and alternative variants of this core set are proposed. This is done to refine rationality postulates and make them usable in different application contexts.

This is also the case in the context of inconsistency measures for non-monotonic logics, for which \cite{ulbricht_measuring_2018} and \cite{ulbricht_handling_2020} make several revisions to existing rationality postulates. The focal point here is on the monotony postulate which requires \(\mathcal{I}(\mathcal{K}) \leq \mathcal{I}(\mathcal{K}')\). In contrast to classical inconsistency measures, however, this postulate does not work for non-monotonic frameworks, as new information can resolve inconsistencies in a knowledge base, which means that \(\mathcal{I}(\mathcal{K}') < \mathcal{I}(\mathcal{K})\) is not only possible but also expected.

\subsection{Basic Postulates}
There are four basic postulates - Consistency, Monotony, Free Formula Independence, and Dominance - which a basic inconsistency measure should have, according to \cite{hunter_measure_2010}. Based on these, \cite{ulbricht_handling_2020} phrases the following basic postulates for non-monotonic logics.
\\
\textbf{Consistency}
For any knowledge base \(\mathcal{K} \subseteq \text{WF}\), \(\mathcal{I}(\mathcal{K}) = 0\) if and only if \(\mathcal{K}\) is consistent.

This requires an inconsistency measure to be able to distinguish between consistent and inconsistent knowledge bases and is true for non-monotonic logics without any modifications.
\\
\textit{\textbf{Monotonicity}}
If \(\mathcal{K}\) and \(\mathcal{K}'\) are knowledge bases, then \(\mathcal{I}(\mathcal{K}) \leq \mathcal{I}(\mathcal{K \cup \mathcal{K}'})\).

Monotonicity is a usually widely accepted postulate, which describes the intuition that a superset of a knowledge base cannot reduce inconsistency, only increase it, since more information is likely to be accompanied by more conflicts.

For non-monotonic logics, however, monotonicity does not make sense as new information might resolve conflicts in \(\mathcal{K}\). Monotonic behavior should only appear if \(\mathcal{K}'\) does not resolve any conflicts in \(\mathcal{K}\). If \(\mathcal{H} \subseteq \mathcal{K}\) is strongly inconsistent, so \(\mathcal{H} \in \text{SI}(\mathcal{K})\), then there should not exist a \(\mathcal{H}' \subseteq \mathcal{K}'\), such that \(\mathcal{H} \cup \mathcal{H}'\) is consistent. If there was, \(\mathcal{H}\) might not contribute to \(\mathcal{I}(\mathcal{K} \cup \mathcal{K}')\), which in turn questions the comparability of \(\mathcal{I}(\mathcal{K})\) and \(\mathcal{I}(\mathcal{K} \cup \mathcal{K}')\).


Consider the following example as an illustration:
[Example here]

This results in the need for a rephrased monotonicity postulate that recognizes the preservation of conflicts as a condition for monotonicity in non-monotonic logics. Using the concept of strongly inconsistent subsets, preserving conflicts is defined as follows:

\begin{definition}
    Let \(\mathcal{K}\) and \(\mathcal{K}'\) be knowledge bases. \(\mathcal{K}'\) preserves conflicts of \(\mathcal{K}\) if \(\mathcal{H} \in \text{SI}(\mathcal{K} \cup \mathcal{K}')\) for any \(\mathcal{H} \in \text{SI}(\mathcal{K})\).
\end{definition}

Applying this concept of conflict preservation, \cite{ulbricht_handling_2020} provides the rephrased monotonicity postulate under the name strong monotonicity.
\\
\textbf{Strong Monotonicity}
If \(\mathcal{K}'\) preserves conflicts of \(\mathcal{K}\), then \(\mathcal{I}(\mathcal{K}) \leq \mathcal{I}(\mathcal{K \cup \mathcal{K}'})\).

Now coming to the free formula independence postulate which states that a free formula \(\alpha\) of a knowledge base \(\mathcal{K}\) should not increase the degree of inconsistency of \(\mathcal{K}\) regardless of its existence, so \(\mathcal{I}(\mathcal{K}) = \mathcal{I}(\mathcal{K} \backslash \{\alpha\})\) for \(\alpha \in Free(\mathcal{K})\). Free in this context then means that the formula \(\alpha\) is not part of a minimal inconsistent subset \(I_{\min}(\mathcal{K})\).

\begin{definition}
    Let \(\mathcal{K}\) be a monotonic knowledge base. A formula \(\alpha \in \mathcal{K}\) is called free if\\
    \(\alpha \in \mathcal{K} \backslash \bigcup\limits_{\mathcal{H} \in I_{\min}(\mathcal{K})} \mathcal{H}\).\\
    Let \(Free(\mathcal{K})\) be the set of all free formulas of \(\mathcal{K}\).
\end{definition}

To generalize the definition of free to non-monotonic logics as well, the concepts of hitting set duality and strong inconsistency can be applied. This results in the minimal inconsistent subset \(I_{\min}(\mathcal{K})\) to be replaced with \(SI_{\min}(\mathcal{K})\) and instead of \(\alpha\) having to appear in any minimal hitting set of \(SI_{\min}(\mathcal{K})\) one can assume \(\alpha\) to be present in all maximal consistent sets \(\mathcal{H} \in C_{\max}(\mathcal{K})\). This leads to the following definition:

\begin{definition}
    Let \(\mathcal{K}\) be a knowledge base. A formula \(\alpha \in \mathcal{K}\) is called free with respect to strong inconsistency (or SI-free or simply free if there is no risk of confusion) if\\
    \(\alpha \in \mathcal{K} \backslash \bigcup\limits_{\mathcal{H} \in SI_{\min}(\mathcal{K})} \mathcal{H} = \bigcap\limits_{\mathcal{H} \in C_{\max}(\mathcal{K})} \mathcal{H}\).\\
    Let \(Free_{SI}(\mathcal{K})\) be the set of all SI-free formulas of \(\mathcal{K}\).
\end{definition}

[Example here]

Given the similarities between \(Free(\mathcal{K})\) and \(Free_{SI}(\mathcal{K})\), a revised postulate for free formula independence is proposed by \cite{ulbricht_handling_2020}. This SI-Free postulate suggests that a free formula \(\alpha\) not only does not increase the degree of inconsistency but also does not introduce strongly inconsistent subsets.
\\
\textbf{SI-Free}
If \(\alpha \in Free_{SI}(\mathcal{K})\), then \(\mathcal{I}(\mathcal{K}) \leq \mathcal{I}(\mathcal{K} \backslash \{\alpha\})\).

One problem in using this postulate stems from the fact that ordinary inconsistency of a subset \(\mathcal{H} \subseteq \mathcal{K}\) only depends on \(\mathcal{H}\), whereas strong inconsistency relates to the whole knowledgebase base \(\mathcal{K}\). The removal of formulas in \(Free_{SI}(\mathcal{K})\) may affect the structure of \(SI_{\min}(\mathcal{K})\).

    [Example here]

This shows that it is hard to predict what might happen when an SI-free formula is removed. Notably, \(SI_{\min}(\mathcal{K}) = SI_{\min}(\mathcal{K} \backslash \{\alpha\})\) is not true for all \(\alpha \in Free_{SI}(\mathcal{K})\), which shows that not even previously introduced \(\mathcal{I}_{\text{MSI}}\) satisfies this SI-free postulate. In non-monotonic logics, an SI-free formula \(\alpha\) could resolve conflicts, which is why \cite{brewka_strong_2017} introduces the notion of neutral formulas to further strengthen the condition.

\begin{definition}
    Let \(\mathcal{K}\) be a knowledge base. A formula \(\alpha \in \mathcal{K}\) is called neutral if it satisfies\\
    \(\forall \mathcal{H} \subseteq \mathcal{K}: \mathcal{H} \in C(\mathcal{K}) \Leftrightarrow \mathcal{H} \cup \{\alpha\} \in C(\mathcal{K})\).\\
    Let \(Ntr(\mathcal{K})\) be the set of all neutral formulas of \(\mathcal{K}\).
\end{definition}

Neutral formulas do not make use of strong inconsistency, in contrast to SI-free formulas. Since neutral formulas do not depend as much on the structure of the knowledge base, they do not influence \(\mathcal{K}\) or \(SI_{\min}(\mathcal{K})\). Like SI-free formulas, a neutral formula \(\alpha \in \mathcal{K}\) does not increase the degree of inconsistency of \(\mathcal{K}\) but also does not resolve any conflicts either, which leads to the independence postulate.
\\
\textbf{Independence}
If \(\alpha \in Ntr(\mathcal{K})\), then \(\mathcal{I}(\mathcal{K}) = \mathcal{I}(\mathcal{K} \backslash \{\alpha\})\).

The only postulate not mentioned yet is the dominance postulate \cite{hunter_measure_2010}. It tries to formulate the intuition that a formula carrying more information is also more likely to be involved in conflicts. It is highly contested, even for propositional settings, see \cite{ferme_revisiting_2014}, and \cite{ulbricht_measuring_2018} questions its use altogether for non-monotonic settings. Since for non-monotonic logics more information can even resolve conflicts in a knowledge base it is unfounded to consider such a formula to be more problematic.

This concludes the discussion on basic postulates.

\subsection{Extended Postulates}
"Many concrete approaches to inconsistency measurement depend on the syntax of a knowledge base. The most common example is the difference between the conjunction \(\{a \land b\}\) and two formulas \(\{a, b\}\)."
"There are thus rationality postulates in the literature that are concerned about the behavior of inconsistency measures when dealing with equivalent formulas resp. equivalent knowledge bases. Of course, it is desirable that a measure \(\mathcal{I}\) is robust wrt. the syntax of \(\mathcal{K}\)."
"The postulate adjunction invariance \cite{ferme_revisiting_2014} formalizes the idea that there should be no difference between \(\{a \land b\}\) and \(\{a, b\}\), i.e., \(\mathcal{I}(\mathcal{K} \cup \{a \land b\}) = \mathcal{I}(\mathcal{K} \cup \{a, b\})\)."
"In non-monotonic frameworks, a notion of equivalence of the form “\(\mathcal{K}\) has the same models as \(\mathcal{K}'\)” is too weak as conclusions can be withdrawn due to non-monotonicity. This observation has led to the development of strong equivalence \cite{lifschitz_strongly_2001}. Strong equivalence can be generalized to arbitrary logics in the following way \cite{brewka_strong_2019}:"

\begin{definition}
    Let \(\mathcal{L} = (\text{WF, BS, INC, ACC})\) be a logic. The knowledge bases \(\mathcal{K}\) and \(\mathcal{K}'\) are strongly equivalent, denoted by \(\mathcal{K} \equiv_S  \mathcal{K}'\), if \(ACC(\mathcal{K} \cup \mathcal{G}) = ACC(\mathcal{K}' \cup \mathcal{G})\) for each knowledge base \(\mathcal{G}\).
\end{definition}
% \\
\textbf{Strong Equivalence}
If \(\mathcal{K} \equiv_S \mathcal{K}'\), then \(\mathcal{I}(\mathcal{K}) = \mathcal{I}(\mathcal{K}')\).
\\
\textbf{FW-Strong Equivalence}
If \(\mathcal{K} \equiv_{\alpha} \mathcal{K}'\), then \(\mathcal{I}(\mathcal{K}) = \mathcal{I}(\mathcal{K}')\).
\\
\textbf{Strong Equivalence Replacement}
If \(\{\alpha\} \equiv_S \{\alpha'\}\) and \(\alpha \notin \mathcal{K}\) as well as \(\alpha' \notin \mathcal{K}\), then \(\mathcal{I}(\mathcal{K} \cup \{\alpha\}) = \mathcal{I}(\mathcal{K} \cup \{\alpha'\})\).
\\
\textbf{Separability}
If \(SI_{\min}(\mathcal{K} \cup \mathcal{K}') = SI_{\min}(\mathcal{K}) \cup SI_{\min}(\mathcal{K}')\) and \(SI_{\min}(\mathcal{K}) \cap SI_{\min}(\mathcal{K}') = \emptyset\), then \(\mathcal{I}(\mathcal{K} \cup \mathcal{K}') = \mathcal{I}(\mathcal{K}) + \mathcal{I}(\mathcal{K}')\).
\\
\textbf{Strong Super-Additivity}
If \(\mathcal{K}\) and \(\mathcal{K}'\) preserve each others's conflicts and \(\mathcal{K} \cap \mathcal{K}' = \emptyset\), then \(\mathcal{I}(\mathcal{K}) + \mathcal{I}(\mathcal{K'}) \leq \mathcal{I}(\mathcal{K} \cup \mathcal{K}')\).

\section{Evaluation of Measures}
