\section{Introduction}
text

\section{Background}
To establish results that apply to any form of logic, it is essential to first define what is meant by "logic" in a formal context.
Logic is generally characterized by two main components: syntax and semantics.
Syntax pertains to the structure and formation of well-formed formulas that can be included in knowledge bases.
Semantics, on the other hand, involve the underlying consequence relations that give meaning to these formulas.

In order to address logics that produce multiple sets of consequences, often referred to as belief sets, it is necessary to link a single knowledge base to several belief sets.
This is crucial for capturing the diversity of possible outcomes that can arise from different interpretations or applications of the same knowledge base.
Moreover, there are instances where the syntax used in knowledge bases may differ from that used in belief sets.
In such cases, it is important to clearly specify what constitutes a belief set to ensure consistency and coherence in logical reasoning.

\cite{brewka_equilibria_2007} states that a logic \(L\) constitutes a set \(KB_L\) of knowledge bases, a set \(BS_L\) of belief sets, and an acceptability function \(ACC_L: KB_L \to 2^{BS_L}\), that maps knowledge bases to sets of belief set.
In this paper knowledge bases are considered sets of formulas, and the elements of belief sets are called beliefs.
The distinction between consistent and inconsistent belief sets is crucial.
By defining inconsistency abstractly and independently of the specific syntax of belief sets (such as the presence of negation), one can identify which belief sets are inconsistent.
Based on this definition \cite{brewka_strong_2019} proposes an extension to \cite{brewka_equilibria_2007}:

A logic \(L\) is a tuple \(L = (WF, BS, INC, ACC)\), where:
\begin{itemize}
      \item \(WF\) is the set of well-formed formulas.
      \item \(BS\) is the set of belief sets.
      \item \(INC \subseteq BS\) is an upward-closed set of inconsistent belief sets.
      \item \(ACC: KB \to 2^{BS}\) assigns a collection of belief sets to each subset of \(WF\).
\end{itemize}

A knowledge base \(K\) in logic \(L\) is a finite subset of \(WF\).
\(K\) is considered inconsistent if all of its belief sets are inconsistent, including cases where \(K\) has no belief sets at all.
A knowledge base is consistent if at least one of its belief sets is not in \(INC\).
When it is stated that a particular instance \(L = (WF, BS, INC, ACC)\) "properly models" a logic \(L\), it implies:
\begin{itemize}
      \item \(WF\) corresponds to the well-formed formulas of \(L\).
      \item \(BS\) includes all sets of beliefs that any knowledge base of \(L\) might have.
      \item \(INC\) contains the belief sets that are considered inconsistent in \(L\).
      \item \(ACC\) assigns the correct belief sets to each knowledge base \(K\) according to \(L\).
\end{itemize}

In the following sections, examples of this definition applied to different logical systems are provided, such as propositional logic, quantified Boolean formulas, answer set programming, and abstract argumentation frameworks. These examples are to demonstrate the flexibility and generality of the definition.

\subsection{Propositional logic}
Denoting a propositional logic as proposed by \cite{gelfond_classical_1991} involves a set of propositional atoms, denoted as \(A\).
Well-formed formulas, denoted in regards to \(A\) as \(WF^P_A\), are created using these atoms with logical operators like negation (\(\neg\)), conjunction (\(\land\)), and disjunction (\(\lor\)).
The classical entailment relation, denoted as \(\models\), states that a formula \(\phi\) is entailed by a set \(K\) of formulas if \(\phi\) is true in every model where all formulas in \(K\) are true.
The set of all deductively closed sets of \(WF^P_A\) forms the belief sets (\(BS^P_A\)) in propositional logic.
The acceptability function (\(ACC^P_A\)) assigns to each knowledge base \(K\) its deductive closure.
The only inconsistent belief set is the set of \(WF^P_A\), which defines \(INC^P_A\).

If for example a propositional signature \(A_1 = \{a,b\}\) with the knowledge bases
\[K_1 = \{a, a \Rightarrow b\} \quad K_2 = \{a, \neg a\}\]
was given.
Then for \(K_1\) the belief set includes all logical consequences of \(a\) and \(a \Rightarrow b\), such as \(b\).
\(K_2\), however, is inconsistent because it contains both \(a\) and \(\neg a\).

\subsection{Quantified Boolean Formulas}
Quantified Boolean Formulas (QBFs) as described in \cite{gelfond_logic_2002} extend propositional logic by introducing quantifiers like \(\forall\) and \(\exists\) over variables.
A QBF is true if the formula evaluates to true under the given quantifiers.
For instance, \(\exists x_1 \forall x_2 (x_1 \lor \neg x_2)\) is true if there exists and assignment to \(x_1\) such that for every \(x_2\), the formula \(x_1 \lor \neg x_2\) holds.

If for example \(Q = \{\exists,\forall\}\) and \(X = \{x_1,x_2\}\) with the knowledge bases
\[K_1 = \{x_1 \lor  x_2, x_1 \lor \neg x_2\} \quad K_2 = \{x_1 \Rightarrow x_2, x_2 \Rightarrow x_1\}\]
were given.
The QBF \(\exists x_1 \forall x_2 (x_1 \lor x_2)\) evaluates to true for \(K_1\), while \(\exists x_1 \forall x_2 (x_1 \Rightarrow x_2) \land (x_2 \Rightarrow x_1)\) does not for \(K_2\), indicating inconsistency.

\subsection{Answer Set Programming}
Answer Set Programming, as in \cite{brewka_answer_2011}, is a form of declarative programming oriented towards difficult search problems.
It uses disjunctive logic programs, which consist of rules with a head and a body.
An answer set is a minimal model satisfying the rules of the program.
A program is inconsistent if it has no answer sets or if all answer sets are inconsistent.

If for example the signature \(A = \{a,b\}\) with the logic programs
\[P_1 = \{a., b., \neg b \leftarrow not \; a.\} \quad P_2 = \{b., \neg b \leftarrow not \; a.\}\]
was given.
Here \(P_1\) is consistent because it allows for a valid answer set where \(a\) and \(b\) can both be true without conflict, avoiding any contradictions.
Conversely \(P_2\) is inconsistent because it leads to a contradiction where \(b\) and \(\neg b\) cannot be true simultaneously, resulting in no valid answer sets.

\subsection{Abstract argumentation frameworks}
\cite{dung_acceptability_1995}

\section{Non-Monotonic Logics}
The term "non-monotonic logic" may initially evoke discomfort due to the complexities and deviations from classical logic.
However, the real world often presents scenarios that classical logic cannot adequately address, necessitating the exploration of non-monotonic logic.
In classical logic, the reasoning processes require thoughts to be disciplined and conform to formal rules.
Everything has a distinct meaning, and relationships can be formalized. Based on these assumptions, one can draw irrefutable conclusions from given knowledge using deduction.
This deductive process underpins the monotonicity of classical logic.

In contrast, real-world scenarios are characterized by unexpected events and complex relationships that often resist formalization.
Descriptions of predicates such as "small" are context-dependent; for instance, an elephant is small compared to a mountain but not to a mouse.
Additionally, temporal changes affect the state of objects and systems continuously, highlighting the limitations of classical logic in representing dynamic real-world knowledge.

Human knowledge is inherently incomplete, and everyday life involves drawing conclusions based on incomplete information.
Errors in assumptions and conclusions are common, yet do not collapse the entire worldview.

Several factors are listed by \cite{beierle_methoden_2019} that contribute to this knowledge imperfection:
\begin{itemize}
      \item \textbf{Unknown Information:} Some information is unknown due to lack of learning, forgetfulness, general unavailability, or inaccessibility.
            Time constraints also necessitate decisions based on limited information.
      \item \textbf{Incomplete Descriptions:} Describing real situations completely in a logical sense is impossible.
            Listing all conceivable exceptions to a rule is impractical, and checking all details is time-consuming.
      \item \textbf{Abstraction:} Abstraction from seemingly unimportant aspects to focus on essentials often leads to incomplete information, which can be both sensible and intentional.
      \item \textbf{Situational Characteristics:} Situational characteristics can be overlooked or misinterpreted.
      \item \textbf{Speculation:} Future events can only be speculated upon.
      \item \textbf{Natural language:} Natural language is context-dependent and rarely entirely clear, requiring interpretation and posing a risk of misunderstandings.
\end{itemize}

Despite these challenges, humans handle incomplete information successfully for decision-making and navigation.
This capability is considered a significant intelligence achievement, demonstrating the effectiveness of non-monotonic reasoning.
Humans excel in non-monotonic reasoning, which should be considered a standard form of logic rather than a peripheral one.
However, non-monotonic logic often appears difficult and esoteric due to its complex nature.

\subsection{Theoretical Foundation and Motivation}
In \cite{mcdermott_non-monotonic_1980} the authors present a detailed exploration of non-monotonic logical systems, which are distinguished by the property that new axioms can invalidate previously established theorems.
These logics are essential for modeling the beliefs of active processes operating with incomplete information, requiring assumptions that may need to be revised as new observations are made.
This, for the time, new proposition is regarded as the basis and motivation for modern non-monotonic logics, developing model and proof theories, and offering a proof procedure along with applications for a specific non-monotonic logic.
They demonstrate the completeness of the non-monotonic predicate calculus and the decidability of the non-monotonic sentential calculus, emphasizing the importance of these logics in handling the dynamic nature of real-world information and beliefs.

A critical aspect of non-monotonic logic discussed in \cite{mcdermott_non-monotonic_1980} is its ability to handle routine revisions, where beliefs expressed as universally true have exceptions that must be managed.
This flexibility contrasts with the rigidity of classical logic, which lacks tools for revising formal theories in the face of new, inconsistent information.
For example, the belief that "all animals with beaks are birds" can be contradicted by introducing the platypus.
The primary challenge in revising beliefs lies in non-monotonic logic's ability to incorporate exceptions and adjustments, thereby better reflecting the inherently non-monotonic nature of human reasoning processes.

Non-monotonic inference rules aim not to establish certain knowledge where none exists, but to guide the selection of provisional beliefs, fostering fruitful investigations and informed guesses.
These rules are not intended to independently derive valid conclusions as classical monotonic rules do.
Instead, they help formulate a coherent set of beliefs that, even if later proven incorrect, provide a meaningful framework for further inquiry.

\subsection{Understanding Inconsistencies in Logic Systems}
Strong inconsistency \cite{brewka_strong_2019}

\subsection{Challenges Specific to Non-Monotonic Logics}
Strong equivalency \cite{lifschitz_strongly_2001}
"We mention strong equivalence here not because it is closely related to inconsistency, but because it helps to illustrate what the goals of this paper are. The observation that standard equivalence is not a very useful concept when it comes to nonmonotonic reasoning has led to a whole body of interesting work investigating stronger notions of equivalence. Similarly, starting from the observation that minimal inconsistent subsets are not very useful for nonmonotonic reasoning, we aim for a stronger notion which coincides with the standard notion for classical logics, but is a lot more useful for nonmonotonic logics." \cite{brewka_strong_2019}

\section{Measuring Inconsistency}
Analyze the distribution of inconsistency within a knowledge base using the Shapely measure \cite{hunter_measure_2010} in the classical case.
Measures based on minimal inconsistent sets of a knowledge base \(\mathcal{K}\) \cite{jabbour_mis_2016} were proposed by \cite{ulbricht_measuring_2018}.

\subsection{Measures of Inconsistency in Non-Monotonic logics}
text

\subsection{Case Studies and Applications}
text

\section{Future Directions}
text
